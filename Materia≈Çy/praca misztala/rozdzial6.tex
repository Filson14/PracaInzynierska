\chapter{Architektura projektu}

Opisane w poprzednim rozdziale komponenty łączą się w całość tworząc integralną całość. Aplikacja wykrywająca ruch przesyła dane na serwer. Serwer przetwarza dane i zapisuje je w bazie danych; umożliwia także wgląd do danych poprzez prostą stronę www. 

\begin{figure}[h]
		\centering
		\includegraphics[width=0.34\textwidth,height=0.34\textheight,keepaspectratio]{arch}
		\caption{Architektura systemu}
		\label{fig:arch}
	\end{figure}
	
Elementem leżącym najniżej w architekturze projektu jest baza danych. Do bazy danych bezpośredni dostęp ma jedynie aplikacja serwerowa która łączy się z nią za pomocą biblioteki {\it Hibernate}. Aplikacja serwerowa udostępnia API przy pomocy którego aplikacja wykrywająca ruch przesyła dane na serwer. Dzięki temu API możliwe również jest działanie serwisu www, który pobiera dzięki niemu dane dla użytkownika. 
	
	
\section{Baza danych}
\label{sec:mysql2}

W pracy użyto relacyjnej bazy danych MySQL \cite{Mysql}. W sekcji \ref{sec:dbsql} załączono skrypt którego użyto do założenia bazy danych. Poniżej przedstawiony jest schemat użytej w systemie bazy danych:

	\begin{figure}[p]
		\centering
		\includegraphics[angle=90,width=\textwidth,height=\textheight,keepaspectratio]{database2}
		\caption{Baza danych użyta na serwerze}
		\label{fig:database}
	\end{figure}

\FloatBarrier

