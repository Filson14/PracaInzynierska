\chapter{Implementacja}
\label{cha:implementacja}

Rozdział ten traktuje o tym w jaki sposób praktycznie zaimplementowano wszystkie funkcjonalności systemu - zostały przedstawione tu kluczowe elementy systemu niezbędne do jego działania. 

\section{Serwer}
\label{sec:serwer}

Sekcja ta dotyczy aplikacji serwerowej, a więc elementu systemu, który pobiera dane od detektorów, odpowiednio je przechowuje oraz umożliwia ich pobieranie upoważnionym użytkownikom.

\subsection{RESTful API}
\label{sec:api}

Aplikacja serwerowa wystawia REST-owe API, które umożliwia pełną obsługę systemu. Cechuje się możliwością:

\begin{itemize}
	\item rejestracji nowego użytkownika,
	\item logowania z rozpoznaniem typu aplikacji która wykonuje logowanie (detektor, czy np. klient www),
	\item wylogowania (kończenia sesji),
	\item powiadomienia serwera przez detektor o fakcie wykrywania ruchu ,
	\item ustawienia nazwy dla detektora ruchu celem łatwiejszego rozpoznawania urządzenia przez użytkownika,
	\item wysyłania zdjęć przez detektor na serwer ,
	\item pobrania wszystkich urządzeń podłączonych pod użytkownika,
	\item sprawdzenia czy dane urządzenie aktualnie pracuje, jest online i wykrywa ruch,
	\item pobrania zdjęć wysłanych przez dane urządzenie.
\end{itemize}


W sekcji \ref{sec:serwerApi} umieszczony jest fragment kodu bezpośrednio odpowiadający za udostępnienie usług webowych. Aplikacja została napisana w języku Java z wykorzystaniem biblioteki \cite{Jersey}. Do podłączenia się do bazy danych frameworka Hibernate \cite{Hibernate}

Poniżej znajduje się tabela przedstawiająca typy zapytań, zwracane wartości, oraz potrzebne argumenty do ich wywołania. 

\begin{description}
	\item[Użytkownicy] \hfill \\
	
		\begin{tabularx}{\linewidth} {|c|X|}
		\hline
						& {\bf POST}\\ \hline
		  URL 			& /users/register\\
		  Zwracany typ 	& application/json\\
		  Nagłówek 		& {\bf api\_key}: string\\
		  Parametry 	& {\bf email}: string, {\bf password}: string [SHA-256]\\
		  Opis 			& Rejestruje w systemie nowego użytkownia \\
		  SUKCES 		& {\bf 200}\\
		  PORAŻKA 		& {\bf 401}: zły API key, {\bf 406} użytkownik o podanym adresie email istnieje, lub podano nieprawidłowe dane, {\bf ErrorJSON}\\
		  \hline
		\end{tabularx}
		
		\begin{tabularx}{\linewidth} {|c|X|}
		\hline
						& {\bf POST}\\ \hline
		  URL 			& /users/login\\
		  Zwracany typ 	& application/json\\
		  Nagłówek 		& {\bf api\_key}: string\\
		  Parametry 	& {\bf email}: string, {\bf password}: string [SHA-256]\\
		  Opis 			& Otwiera nową sesję dla aplikacji klienckiej (wyświetlającej dane) \\
		  SUKCES 		& {\bf 200}: \{"Token": string\} \\
		  PORAŻKA 		& {\bf 401}: zły API key, {\bf 406} niepoprawne dane logowania, {\bf ErrorJSON}\\
		  \hline
		\end{tabularx}
		
		\begin{tabularx}{\linewidth} {|c|X|}
			\hline
						& {\bf DELETE}\\ \hline
		  URL 			& /users/logout\\
		  Zwracany typ 	& Kod odpowiedzi HTTP\\
		  Nagłówek 		& {\bf api\_key}: string, {\bf token}: string\\
		  Parametry 	& \\
		  Opis 			& Zamyka sesje \\
		  SUKCES 		& {\bf 200} \\
		  PORAŻKA 		& {\bf 401}: zły API key lub niewłaściwy token\\
		  \hline
		\end{tabularx}
		
		\item[API dla detektora ruchu] \hfill \\
		
		\begin{tabularx}{\linewidth} {|c|X|}
		\hline
						& {\bf POST}\\ \hline
		  URL 			& /detector/login\\
		  Zwracany typ 	& application/json\\
		  Nagłówek 		& {\bf api\_key}: string\\
		  Parametry 	& {\bf email}: string, {\bf password}: string [SHA-256], {\bf device\_id}: string\\
		  Opis 			& Otwiera sesję dla detektora ruchu. Jeżeli datektor o podanym device\_id nie istnieje tworzony jest odpowiedni wpis w bazie danych \\
		  SUKCES 		& {\bf 200}: \{"Token": string\}\\
		  PORAŻKA 		& {\bf 401}: zły API key, {\bf 406} niepoprawne dane logowania, {\bf ErrorJSON}\\
		  \hline
		\end{tabularx}
		
		\begin{tabularx}{\linewidth} {|c|X|}
		\hline
						& {\bf POST}\\ \hline
		  URL 			& /detector/beep\\
		  Zwracany typ 	& Kod odpowiedzi HTTP\\
		  Nagłówek 		& {\bf api\_key}: string, {\bf token}: string\\
		  Parametry 	& {\bf device\_id}: string\\
		  Opis 			& Dodaje wpis o tym, że detektor działa\\
		  SUKCES 		& {\bf 200}\\
		  PORAŻKA 		& {\bf 401}: zły API key, lub niewłaściwy token, {\bf 404} urządzenie nie znalezione\\
		  \hline
		\end{tabularx}
		
		\begin{tabularx}{\linewidth} {|c|X|}
		\hline
						& {\bf POST}\\ \hline
		  URL 			& /detector/devicename\\
		  Zwracany typ 	& application/json\\
		  Nagłówek 		& {\bf api\_key}: string, {\bf token}: string\\
		  Parametry 	& {\bf device\_id}: string, {\bf name}: string\\
		  Opis 			& Ustawia nazwę dla urządzenia\\
		  SUKCES 		& {\bf 200}\\
		  PORAŻKA 		& {\bf 401}: zły API key, lub niewłaściwy token, {\bf 404} urządzenie nie znalezione\\
		  \hline
		\end{tabularx}
		
		\begin{tabularx}{\linewidth} {|c|X|}
		\hline
						& {\bf POST}\\ \hline
		  URL 			& /detector/photo\\
		  Konsumuje		& mediapart/form-data\\
		  Zwracany typ 	& Kod odpowiedzi HTTP\\
		  Nagłówek 		& {\bf api\_key}: string, {\bf token}: string\\
		  Parametry 	& {\bf device\_id}: string, {\bf photo}: file\\
		  Opis 			& Wysyła zdjęcie oznaczające ruch\\
		  SUKCES 		& {\bf 200}\\
		  PORAŻKA 		& {\bf 401}: zły API key, lub niewłaściwy token, {\bf 404} urządzenie nie znalezione\\
		  \hline
		\end{tabularx}
		
		\item[API dla klienta] \hfill \\
		
		\begin{tabularx}{\linewidth} {|c|X|}
		\hline
						& {\bf GET}\\ \hline
		  URL 			& /client/devices\\
		  Zwracany typ 	& application/json\\
		  Nagłówek 		& {\bf api\_key}: string, {\bf token}: string\\
		  Parametry 	& \\
		  Opis 			& Zwraca listę wszystkich detektorów ruchu użytkownika\\
		  SUKCES 		& {\bf 200}: JSON \{"Devices": [\{"DeviceId": string, "DeviceName": string\}, ... ]\}\\
		  PORAŻKA 		& {\bf 401}: zły API key, lub niewłaściwy token, {\bf 404} użytkownik nie znaleziony\\
		  \hline
		\end{tabularx}
		
		\begin{tabularx}{\linewidth} {|c|X|}
		\hline
						& {\bf GET}\\ \hline
		  URL 			& /client/isonline\\
		  Zwracany typ 	& application/json\\
		  Nagłówek 		& {\bf api\_key}: string, {\bf token}: string\\
		  Parametry 	& {\bf device\_id}: string\\
		  Opis 			& Sprawdza czy detektor jest OnLine\\
		  SUKCES 		& {\bf 200}: JSON \{``Online'': boolean\}\\
		  PORAŻKA 		& {\bf 401}: zły API key, lub niewłaściwy token, {\bf 404} urządzenie nie znalezione\\
		  \hline
		\end{tabularx}
		
		\begin{tabularx}{\linewidth} {|c|X|}
		\hline
						& {\bf GET}\\ \hline
		  URL 			& /client/iamges\\
		  Zwracany typ 	& application/json\\
		  Nagłówek 		& {\bf api\_key}: string, {\bf token}: string\\
		  Parametry 	& {\bf device\_id}: string\\
		  Opis 			& Zwraca listę zdjęć wysłanych przez urządzenie \\
		  SUKCES 		& {\bf 200}: JSON \{"Images": [\{"Id": integer, "Date": string\}, ...]\}, gdzie Id oznacza Id zdjęcia w strumieniu zdjęć \\
		  PORAŻKA 		& {\bf 401}: zły API key, lub niewłaściwy token, {\bf 404} urządzenie nie znalezione, ErrorJSON\\
		  \hline
		\end{tabularx}
		
		\begin{tabularx}{\linewidth} {|c|X|}
		\hline
						& {\bf GET}\\ \hline
		  URL 			&  /client/fetch/\{id\} \\ 
  		  Zwracany typ 	& image/jpg\\
		  Nagłówek 		& {\bf api\_key}: string, {\bf token}: string\\
		  Parametry 	& {\bf device\_id}: string, {\bf \{id\}}: integer\\
		  Opis 			& Zwraca zdjęcie o \{id\} wysłane przez urządzenie o danym device\_id\\
		  SUKCES 		& {\bf 200}: image/jpg\\
		  PORAŻKA 		& {\bf 401}: zły API key, lub niewłaściwy token, {\bf 404} urządzenie nie znalezione, lub zdjęcie nie znalezione, ErrorJSON\\

		  \hline
		\end{tabularx}

\end{description}

Poniżej przedstawiona jest forma wiadomości ErrorJSON:

\begin{lstlisting}
ErrorJSON:
	{
		"OperationType": string, 
		"ErrorMessage": string
	}
\end{lstlisting}
	
\section{Aplikacja}
\label{sec:aplikacja}

Do implementacji aplikacji na system Android użyto IDE Android Studio w wersji 1.0.0, oraz systemu do automatycznego budowania aplikacji Gradle. Do działania wymaga wersji 4.0 lub wyższej systemu Android. Najważniejsze fragmenty kodu załączono w sekcji \ref{sec:android}. 

Aplikacja cechuje się następującymi funkcjonalnościami: 
\begin{itemize}
	\item wykrywania ruchu oraz prezentacji jego wyników w trybie: kolorowym, odcieniu szarości, oraz binarnym,
	\item łączenia się ze zdalnym serwerem, oraz przesyłania mu informacji o tym czy wykrywanie jest włączone,
	\item wysyłania na serwer zdjęć zawierających klatki z wykrytym ruchem,
	\item zaznaczania, oraz wycinania obszaru ruchu z klatki, oraz jego zapisu na kartę pamięci,
	\item włączenia alarmu głosowego w razie wykrytego ruchu,
	\item ustalenia parametrów algorytmu, które zostały opisane w sekcji \ref{sec:użyty algorytm}.

\end{itemize}

Aplikacja do wykonania obliczeń na obrazach wykorzystuje bibliotekę OpenCV, napisaną w C++. Dla zwiększenia OpenCV potrzebuje bibliotek napisanych pod konkretny hardware. Z tego względu wymagane jest pobranie dodatkowej aplikacji. Proces ten wykonywany jest przy pierwszym uruchomieniu detektora ruchu.

Po naciśnięciu przycisku detektor rozpoczyna analizę kolejnych klatek w czasie rzeczywistym, zgodnie z algorytmem opisanym w sekcji \ref{sec:użyty algorytm}. Proces ten jest jednowątkowy, mimo to nawet na słabszych urządzeniach otrzymano zadowalające rezultaty 13fps. 

Poniżej przedstawiono zrzuty ekranu:

\begin{figure}
\centering
	\begin{subfigure}{.5\textwidth}
	  \centering
	  \includegraphics[width=.7\linewidth]{scr1}
	  \caption{Wykrywanie ruchu w trybie RGB}
	  \label{fig:sub1}
	\end{subfigure}%
	\begin{subfigure}{.5\textwidth}
	  \centering
	  \includegraphics[width=.7\linewidth]{scr2}
	  \caption{Wykrywanie ruchu w trybie GRAY}
	  \label{fig:sub2}
	\end{subfigure}
	
	\begin{subfigure}{.5\textwidth}
	  \centering
	  \includegraphics[width=.7\linewidth]{scr3}
	  \caption{Wykrywanie ruchu w trybie binarnym}
	  \label{fig:scr3}
	\end{subfigure}%
	\begin{subfigure}{.5\textwidth}
	  \centering
	  \includegraphics[width=.7\linewidth]{scr4}
	  \caption{Zrzut ekranu części opcji programu}
	  \label{fig:scr4}
	\end{subfigure}
\caption{Zrzuty ekrany działającej aplikacji}
\label{fig:test}
\end{figure}

\FloatBarrier
\subsection{Użyte biblioteki zewnętrzne}
\label{sec:biblioteki zewnętrzne}

Przy tworzeniu aplikacji wykrywającej ruch na system Android użyto kilku bibliotek, jeżeli nie napisano inaczej udostępnionych na licencji Apache 2.0 \cite{Apachelicense}. Oto one:

\begin{description}
	\item[OPEN CV] \cite{Opencv}, wersja 2.4.9 \hfil \\
	{\it Open Source Computer Vision} jako jedyna z wykorzystanych bibliotek udostępniona na licencji BSD. OpenCV udostępnia wiele narzędzi z zakresu przetwarzania obrazów. Została napisana przy użyciu Android NDK w języku C++ dzięki czemu działa ona znacznie szybciej. OpenCV wymaga zainstalowania na Smartfonie binarnych plików dla konkretnego hardware'u, jednak aplikacja stworzona w ramach tej pracy przeprowadza użytkownika przez cały proces. 
	
	\item[Retrofit] \cite{Retrofit}, wersja 1.8.0 \hfil \\
	Retrofit jest lekkim klientem REST'a dla Androida. Umożliwia zarówno synchroniczne jak asynchroniczne wywołania. Retrofit zamienia RESTful API w interfejs, oraz używa biblioteki GSON do automatycznego parsowania odpowiedzi serwera. Zastosowanie biblioteki załączono w sekcji \ref{sec:androidApi}
	
	\item[GSON] \cite{Gson}, wersja 2.3.1 \hfil \\
	GSON jest biblioteką która automatycznie przekształca String w formie JSON \cite{Json} na obiekt Javowy.
	
	\item[ButterKnife] \cite{Butterknife} \hfil \\
	ButterKnife jest małą biblioteką która za pomocą {\it Java annotations} ułatwia odnajdowanie i programowanie widoków 

\end{description}

