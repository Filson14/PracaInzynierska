\chapter{Wstęp}
\section{Przedmowa}
Natura to nieposkromiona siła, którą człowiek stara się poznać. Choć już od lat 60-tych prowadzone są badania nad systemami modyfikacji pogody (w szczególności do zastosowań militarnych), to prawidłowe przewidywanie zjawisk atmosferycznych, a także ich skutków, wciąż stanowi problem wymagający rozwiązania~\cite{systemy_kontroli_pogody}.

W lecie 2010 roku Polskę zaatakowała ogromna powódź, która dotknęła niemal wszystkich województw. W wyniku wzmożonych opadów rzeki wezbrały i w wielu miejscach wystąpiły z koryta. Mimo istnienia około stu zbiorników retencyjnych oraz wielu zapór wodnych, nie udało się zapobiec tragedii, w~wyniku której mnóstwo ludzi straciło swój dobytek~\cite{powodz}.

Istniejąca obecnie infrastruktura dokonująca pomiarów zjawisk pogodowych czy stanów zbiorników wodnych może stanowić bazę do systemów umożliwiających zapobieganie takim tragediom. Analiza wielkości opadu oraz przewidywanie drogi jej spływu może pozwolić na przykład operatorom zapór wodnych na podjęcie właściwych, a~przede wszystkim odpowiednio wczesnych działań tak, aby przygotować zbiorniki retencyjne na przyjęcie dodatkowej ilości wody oraz w~sposób kontrolowany pokierować jej spływ i zabezpieczyć ludność~\cite{bodziony, imgw}.

Praca ta traktować będzie o~elemencie składowym przybliżonego powyżej rozwiązania, czyli analizie danych opadowych.

\section{Cele pracy}
\label{sec:cele}
Celem pracy jest analiza możliwości uzyskania danych pogodowych, zbudowanie mechanizmu przekształcającego wskazania z~posterunków opadowych na objętość opadu w~zadanym obszarze zorientowanego na obserwację pojedynczego opadu, a~także przeprowadzenie badania wpływu usunięcia poszczególnych posterunków na wartości wynikowe. Zrealizowany zostanie także proces analityki wskazanych pomiarów. 

% Elementy składowe pracy to:
%\begin{itemize}
%\item
%Odczyt z serwisu pogodynka.pl danych posterunków opadowych i wodowskazowych (takich jak nazwa i lokalizacja), wartości odpowiednich wskazań posterunków oraz parametrów granic zlewni.
%\item
%Przekształcenie danych opadowych metodą triangulacji Delaunaya i aplikacja tej metody do przebiegu granic zlewni.
%\item
%Analiza korelacji opadu powierzchniowego z~wysokością poziomu wody na odpowiednich posterunkach wodowskazowych.
%\end{itemize}

\section{Opis dokumentu}

Praca ta porusza tematykę analizy danych opadowych pochodzących z~posterunków opadowych. Opisuje przegląd metod przekształcenia wartości punktowej opadu na wartość powierzchniową w~obszarze wskazanej zlewni rzeki oraz implementację wybranego rozwiązania.



Rozdział~\ref{cha:odczyt_danych} opisuje proces analizy możliwości poboru danych meteorologicznych i~hydrologicznych.



Kolejna część niniejszej pracy przybliża stosowane powszechnie metody wyznaczania opadu powierzchniowego, czynniki wpływające na wybór właściwej dla omawianego problemu oraz sposób jej działania.



\textbf{Implementacja}, to przedstawienie stworzonego programu. Opisuje sposób praktycznego użycia wybranego algorytmu oraz użytych narzędzi.




Dalej, przeprowadzona jest analiza wyników działania programu. Wskazano także wartość błędu jaka może występować.



Na koniec przeprowadzone zostało podsumowanie całej pracy. Przytoczono wnioski wyciągnięte z~jej działania oraz możliwe rozszerzenia.

