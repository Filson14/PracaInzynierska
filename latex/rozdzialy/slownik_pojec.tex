\chapter{Słownik pojęć}
\begin{description}[leftmargin=5cm]
\item[Posterunek opadowy] \hfill \\ 
Miejsce przeprowadzania pomiarów i obserwacji meteorologicznych. Mierzona jest ilość wody opadowej, która zebrała się w~deszczomierzu.
\item[Posterunek wodowskazowy] \hfill \\ 
Miejsce prowadzenia pomiarów stanu wody za pomocą wodowskazu.
\item[Profil wodowskazowy] \hfill \\ 
Punkt na rzece, w~którym zamontowany jest wodowskaz.
\item[Zlewnia] \hfill \\ 
Obszar, z którego wody spływają do jednego punktu danego zbiornika wodnego lub jego fragmentu.
\item[Zbiornik retencyjny] \hfill \\
Sztuczny zbiornik wodny powstały w wyniku zatamowania wód rzecznych przez zaporę. Pełnić może wiele funkcji, takich jak przeciwpowodziowa energetyczna czy nawet rekreacyjna.
\item[Opad] \hfill \\
Mierzony punktowo w~miejscu posterunku opadowego z~użyciem deszczomierzy.
\item[Izohieta] \hfill \\
Linia łącząca punkty o jednakowej wysokości opadu.
\item[Triangulacja] \hfill \\
Metoda łączenia zbioru punktów w sieć trójkątów o~wierzchołkach w~zadanych punktach.
\end{description}