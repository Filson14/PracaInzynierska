\chapter{Słownik pojęć}
\begin{description}[leftmargin=6cm]

\item[Posterunek opadowy]
Miejsce przeprowadzania pomiarów i obserwacji meteorologicznych. Mierzona jest ilość wody opadowej, która zebrała się w~deszczomierzu.

\item[Posterunek wodowskazowy]
Miejsce prowadzenia pomiarów stanu wody za pomocą wodowskazu.

\item[Profil wodowskazowy]
Punkt na rzece, w~którym zamontowany jest wodowskaz.

\item[Zlewnia]
Obszar, z którego wody spływają do jednego punktu danego zbiornika wodnego lub jego fragmentu.

\item[Zbiornik retencyjny]
Sztuczny zbiornik wodny powstały w wyniku zatamowania wód rzecznych przez zaporę. Pełnić może wiele funkcji, takich jak przeciwpowodziowa energetyczna czy nawet rekreacyjna.

%\item[Opad]
%Mierzony punktowo w~miejscu posterunku opadowego z~użyciem deszczomierzy.

\item[Izohieta]
Linia łącząca punkty o~jednakowej wysokości opadu.

\item[Triangulacja]
Zagadnienie geometrii obliczeniowej. Metoda łączenia zbioru punktów w~sieć trójkątów (przestrzeń dwuwymiarowa) bądź czworościanów (przestrzeń trójwymiarowa) o~wierzchołkach w~zadanych punktach. Sposób rozkładu kształtu na mniej skomplikowane elementy.
\end{description}