\chapter{Odczyt danych pogodowych}
W celu odczytu potrzebnych danych napisano funkcje w~środowisku Matlab. Opierają się na zapytaniu HTTP do API dostępnego na stronie monitor.pogodynka.pl i właściwym parsowaniu odpowiedzi.

Dane z serwera odbierane są w formacie JSON. Do jego obsługi w~Matlab'ie zastosowano bibliotekę JSONlab, która zawiera funkcję loadjson(). Samo zapytanie wykonywane jest poprzez funkcję webread().

Funkcja loadjson() przyjmuje łańcuch znaków zawierający opis obiektu JSON. W wyniku jej działania każdy taki obiekt przekształcany jest do struktury danych Matlab zwanej \textbf{cell array} zawierającej indeksowane kontenery mogące przechowywać różne typy danych. Tablice JSON są przekształcane do tablic Matlab. 

Kody źródłowe poszczególnych funkcji znajdują się w~podrozdziale~\ref{sec:kod_odczyt_danych}.
\section{Dane posterunków}
Za pomocą odpowiednich funkcji do pamięci zmiennych środowiska ładowane są informacje na temat posterunków opadowych i wodowskazowych. Zapisywane są tutaj ich współrzędne geograficzne, nazwa oraz identyfikator, który będzie użyty do pobierania pomiarów dla danego posterunku.

\section{Dane zlewni}
Pobierane dane zlewni obejmują nazwy oraz współrzędne geograficzne punktów stanowiących ich granice i są zapisywane w kontenerze cell array. Dla ułatwienia przeprowadzenia analizy dla pojedynczej zlewni zastosowano wydzielenie poszczególnych elementów kontenera do osobnej zmiennej.

\section{Pomiar opadu}
\label{sec:pomiar_opadu}
Została napisana funkcja, która dla zadanych na wejściu posterunków opadowych odczytuje pomiary ze wskazanego na początku rozdziału serwisu. Funkcja ta zwraca strukturę zawierającą poszczególne pomiary.

Pojedynczy element zawiera dane ostatniego pomiaru godzinowego i~dobowego oraz dwie tablice z~godzinowymi pomiarami za ostatnie 48 godzin i dobowymi na przestrzeni tygodnia.

Odczytane dane zostają zapisane automatycznie w~pliku .mat.

\section{Pomiar poziomu wody}
Funkcja odczytująca wskazania posterunków wodowskazowych działa analogicznie jak ta, opisana w podrozdziale~\ref{sec:pomiar_opadu}. Pojedynczy element kontenera wynikowego zawiera aktualne wskazanie poziomu wody (wraz z datą pomiaru), tablicę godzinowych pomiarów poziomu wody za ostatnie 72 godziny oraz tablicę godzinowych pomiarów wielkości przepływu (w jednostce $m^3/s$) za taki sam okres czasu.