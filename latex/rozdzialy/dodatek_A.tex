\chapter{Dodatek A - fragmenty kodów źródłowych}
\label{cha:dodatek_A}
\section{Odczyt danych} % Zastanowić się czy ta część powinna się znaleźć w pracy
\label{sec:kod_odczyt_danych}
Jak zaznaczono w~rozdziale~\ref{cha:odczyt_danych}, wykorzystanie poniższego kodu jest niedozwolone bez uzyskania pisemnej zgody Instytutu Meteorologii i Gospodarki Wodnej.

Odczyt obszarów zlewni.
\begin{lstlisting}
function [ catchments ] = loadCatchments( )
  disp('Odczyt obszarów zlewni został rozpoczęty.');
  catchmentsJson = urlread('http://monitor.pogodynka.pl/api/coordinates/?catchments');
  catchments = loadjson(catchmentsJson);
  save('zlewnie.mat','catchments');
  disp('Obszary zlewni zostały załadowane i zapisane w pliku "zlewnie.mat".');
end
\end{lstlisting}

Odczyt posterunków opadowych.
\begin{lstlisting}
function [ stations ] = loadRainfallStations( )
  disp('Odczyt posterunków opadowych rozpoczęty.');
  stationsJson = urlread('http://monitor.pogodynka.pl/api/map/?category=meteo');
  tempData = loadjson(stationsJson);
  nO = length(tempData);
  stations = [];
  for i = 1:nO,
    stations(i).id = tempData{1, i}.i;
    stations(i).name = tempData{1, i}.n;
    stations(i).longitude = tempData{1, i}.lo;
    stations(i).latitude = tempData{1, i}.la;
  end
  disp('Posterunki opadowe zostały załadowane');
end
\end{lstlisting}

Odczyt punktów wodowskazowych.
\begin{lstlisting}
function [ stations ] = loadWaterstateStations( )
  disp('Odczyt posterunków wodowskazowych rozpoczęty.');
  stationsJson = urlread('http://monitor.pogodynka.pl/api/map/?category=hydro');
  tempData = loadjson(stationsJson);
  nO = length(tempData);
  stations = [];
  for i = 1:nO,
    stations(i).id = tempData{1, i}.i;
    stations(i).name = tempData{1, i}.n;
    stations(i).longitude = tempData{1, i}.lo;
    stations(i).latitude = tempData{1, i}.la;
  end
  disp('Posterunki wodowskazowe zostały załadowane');
end
\end{lstlisting}

Odczyt pomiarów posterunku opadowego.
\begin{lstlisting}
function [ rainfallData ] = readRainfallData( rainfallStations )
  URL = 'http://monitor.pogodynka.pl/api/meteo/?id=';
  stationsAmmount = length(rainfallStations); 
  rainfallData = [];
  for i = 1:stationsAmmount,
    url = strcat(URL,num2str(rainfallStations(i).id));
    jsonData = urlread(url);
    station = loadjson(jsonData);
    disp(station.name);
    rainfallData(i).stationIndex = i;
    rainfallData(i).stationId = station.id;
    rainfallData(i).stationName = station.name; 
    rainfallData(i).hourly = station.status.precip;
    rainfallData(i).daily = station.status.precipDaily;
    rainfallData(i).last48hours = station.hourlyPrecipRecords;
    rainfallData(i).lastWeek = station.dailyPrecipRecords;
  end  
  file_name = strcat('precip_',datestr(now,'yyyy_mm_dd_HH_MM'),'.mat');
  save(file_name,'rainfallData');
  disp(['Dane opadowe zapisane w pliku ', file_name]);
end
\end{lstlisting}


Odczyt pomiarów wodowskazu.
\begin{lstlisting}
function [ waterstateData ] = readWaterstateData( waterstateStations )
  URL = 'http://monitor.pogodynka.pl/api/hydro/?id=';
  stationsAmmount = length(waterstateStations);
  waterstateData = [];
  for i = 1:stationsAmmount,
    url = strcat(URL,num2str(waterstateStations(i).id));
    jsonData = urlread(url);
    station = loadjson(jsonData);
    station.name
    waterstateData(i).stationIndex = i;
    waterstateData(i).stationId = station.id;
    waterstateData(i).stationName = station.name;
    waterstateData(i).value = station.status.currentValue;
    waterstateData(i).measurementDate = station.status.currentDate;
    waterstateData(i).waterstateLast72hours = station.waterStateRecords;
    waterstateData(i).dischargeLast72hours = station.dischargeRecords;
  end
  file_name = strcat('waterstate_',datestr(now,'yyyy_mm_dd_HH_MM'),'.mat');
  save(file_name,'waterstateData');
  disp(['Dane wodowskazowe zapisane w pliku ', file_name]);
end
\end{lstlisting}



%==========================================================
\section{Funkcje opadu}
Funkcja paraboliczna
%------------------------------------------
\begin{lstlisting}
function z = paraboloidalPrecip(x, y)
  global middle_x;
  global middle_y;

  x0 = middle_x;
  y0 = middle_y;

  h = 100;
  alpha = 50e-12; 
  beta = 70e-12;

  z = h * (1 - (alpha * (x-x0).^2 + beta*(y-y0).^2));
  [w,k] = size(z);
  for i=1:w,
    for j=1:k,
      if z(i,j) < 0, z(i,j)=0; end
    end
  end
end
\end{lstlisting}

%-----------------------------------------
Funkcja wymierna
\begin{lstlisting}
function z = rationalPrecip(x, y)
  global middle_x;
  global middle_y;

  x0 = middle_x;  y0 = middle_y;

  h = 100;
  alpha = 50e-12; 
  beta = 200e-12;
  z = h * (1 + alpha .* (x - x0).^2 + beta .* (y - y0).^2).^-1;
end
\end{lstlisting}
%-----------------------------------------
Obliczanie całki podwójnej w obszarze.
\begin{lstlisting}
function precip = areaIntegral(fun,x1,x2,y1d,y2d,y1g,y2g)
  % Obliczanie całki podwójnej w obszarze:
  % x należące do [x1, x2],
  % y zawarte między prostą ograniczającą obszar:
  %        od dołu - przechodzącą przez punkty (x1,y1d) i (x2,y2d),
  %        od góry -                           (x1,y1g) i (x2,y2g)

  ymin = @(x) (y2d-y1d)/(x2-x1)*(x-x1)+y1d;
  ymax = @(x) (y2g-y1g)/(x2-x1)*(x-x1)+y1g;
  precip = integral2(fun,x1,x2,ymin,ymax,'RelTol',1e-6);
end
\end{lstlisting}

%-----------------------------------------
Funkcja wyznaczająca opad rzeczywisty na wskazanym trójkącie.
\begin{lstlisting}
function precip = precipOnTriangle( fun,X,Y )
  %Funkcja oblicza wielkość opadu na powierzchnię trójkąta zadanego
  %współrzędnymi 3 wierzchołków
  % Dane wejściowe:
  % fun - funkcja wyznaczająca wielkość opadu w punkcie (x,y),
  % X - wektor odciętych x wierzchołków trójkąta,
  % Y - wektor rzędnych y wierzchołków trójkąta.
  % Dana wyjściowa:
  % precip - ilość wody, jaka spadła na obszar zadanego trójkąta.

  % Sortowanie wierzchołków wzdłuż x - od lewej do prawej:
  [Xs,IX] = sort(X); Ys = Y(IX);
  precip = 0;
  Tol = sum(abs(Xs))*1e-6; 
  if Xs(3)-Xs(1) < Tol, return, end  % Powierzchnia trójkąta bliska 0
  if Xs(2)-Xs(1) < Tol,
    % Całka obliczana w pojedynczym obszarze normalnym
    if Ys(1) < Ys(2),
      precip = areaIntegral(fun,Xs(1),Xs(3),Ys(1),Ys(3),Ys(2),Ys(3));
    else
      precip = areaIntegral(fun,Xs(1),Xs(3),Ys(2),Ys(3),Ys(1),Ys(3)); 
    end
  else if Xs(3)-Xs(2) < Tol,
    % Całka obliczana w pojedynczym obszarze normalnym
    if Ys(2) < Ys(3)
      precip = areaIntegral(fun,Xs(1),Xs(3),Ys(1),Ys(2),Ys(1),Ys(3));
    else
      precip = areaIntegral(fun,Xs(1),Xs(3),Ys(1),Ys(3),Ys(1),Ys(2));
    end
  else 
    Ym = (Ys(3)-Ys(1))/(Xs(3)-Xs(1))*(Xs(2)-Xs(1))+Ys(1);
    if Ys(2) > Ym           % środkowy wierzchołek u góry
      % Całka obliczana w pierwszym obszarze normalnym
      precip1 = areaIntegral(fun,Xs(1),Xs(2),Ys(1),Ym,Ys(1),Ys(2));    
      % Całka obliczana w drugim obszarze normalnym
      precip2 = areaIntegral(fun,Xs(2),Xs(3),Ym,Ys(3),Ys(2),Ys(3));
    else                    % środkowy wierzchołek na dole
      % Całka obliczana w pierwszym obszarze normalnym
      precip1 = areaIntegral(fun,Xs(1),Xs(2),Ys(1),Ys(2),Ys(1),Ym);    
      % Całka obliczana w drugim obszarze normalnym
      precip2 = areaIntegral(fun,Xs(2),Xs(3),Ys(2),Ys(3),Ym,Ys(3));
    end
    precip = precip1+precip2;
  end
end
\end{lstlisting}








\section{Konwersja współrzędnych}
\begin{lstlisting}
disp('Konwersja współrzędnych na wartości metryczne');
alpha = min(rainfallPoints(:, 2)) + (max(rainfallPoints(:, 2)) - min(rainfallPoints(:, 2)))/2;
R = 6378410;                       % Promień równikowy Ziemi w metrach
r = cos(alpha*2*pi/360) * R;
parallelLength = 2 * pi * r;
xDegree = (parallelLength / 360);  % w metrach
yDegree = 111196.672;              % w metrach

metricMiddleX = geoMiddleX * xDegree; % w metrach
metricMiddleY = geoMiddleY * yDegree; % w metrach
middle_x = metricMiddleX;
middle_y = metricMiddleY;

rainfallPoints = [rainfallPoints(:, 1).*xDegree rainfallPoints(:, 2).*yDegree rainfallPoints(:, 3)];
catchment = [catchment(:,1).*xDegree, catchment(:,2).*yDegree];
\end{lstlisting}


\section{Triangulacja}
\begin{lstlisting}
disp('Triangulacja punktów wejściowych.');
figure
hold on;
ylabel(yAxisLabel)
xlabel(xAxisLabel)
dt = delaunayTriangulation(rainfallPoints(:, 1), rainfallPoints(:, 2));
triplot(dt, '-yo', 'MarkerSize', 2, 'MarkerFaceColor','black','MarkerEdgeColor','black');
%% Rysowanie obszaru zlewni na wykresie
plot(catchment(:,1), catchment(:, 2),'-r');
\end{lstlisting}




\section{Wyznaczanie wektorów normalnych trójkąta}
\begin{lstlisting}
disp('Wyznaczanie wektorów normalnych dla płaszczyzn poszczególnych trójkątów.');
numtri = length(dt.ConnectivityList);
triangleVectors = zeros(numtri, 3);
for i = 1:numtri
triangle = dt.ConnectivityList(i, :);
  A = [rainfallPoints(triangle(1), 1), rainfallPoints(triangle(1), 2), rainfallPoints(triangle(1), 3)];
  B = [rainfallPoints(triangle(2), 1), rainfallPoints(triangle(2), 2), rainfallPoints(triangle(2), 3)];
  C = [rainfallPoints(triangle(3), 1), rainfallPoints(triangle(3), 2), rainfallPoints(triangle(3), 3)];

  AB = [B(1) - A(1), B(2) - A(2), B(3) - A(3)];
  BC = [C(1) - B(1), C(2) - B(2), C(3) - B(3)];
    
  triangleVectors(i,:) = cross(AB, BC);
end
\end{lstlisting}



\section{Interpolacja punktów}

Funkcja wyznaczająca wartość punktu w oparciu o interpolację płaszczyzną.
\begin{lstlisting}
function [ z ] = findValueFromSurface( x, y, v, p )
  %   x, y - współrzędne punktu szukanego
  %   v - wektor normalny płaszczyzny [A B C]
  %   p - punkt należący do płaszczyzny
  if v(3) == 0
    z = 0;
  else
    z = p(3) + (v(1)*(x - p(1)) + v(2)*(y - p(2)))/(-v(3));
  end
end
\end{lstlisting}


Szukanie punktów krytycznych dla wyznaczenia opadu powierzchniowego zlewni.
\begin{lstlisting}
disp('Interpolacja wartości dla węzłów granic zlewni oraz punktów przecięcia granicy zlewni z liniami triangulacji.');
x = catchment(:, 1);
y = catchment(:, 2);
pointsAmmount = length(x);
for i = 1:pointsAmmount
  if i ~= 1
    line = [x(i-1) y(i-1) x(i) y(i)];
    numtri = length(dt.ConnectivityList);
    checkedLines = zeros(0,2);
    for j = 1:numtri
      triangle = dt.ConnectivityList(j, :);

      triangle_vx = [rainfallPoints(triangle(1),1),rainfallPoints(triangle(2),1),rainfallPoints(triangle(3),1)];
      triangle_vy = [rainfallPoints(triangle(1),2),rainfallPoints(triangle(2),2),rainfallPoints(triangle(3),2)];

      if max(inpolygon(x(i), y(i), triangle_vx, triangle_vy)) == 1
        point_value = findValueFromSurface(x(i), y(i), triangleVectors(j,:), rainfallPoints(triangle(1),:));
        borderPoints = [borderPoints; [x(i), y(i), point_value]];
        h = plot(x(i),y(i),'bo','MarkerSize',5,'LineWidth',1);
      end

      for k = 1:3
        v1_number = triangle(k);
        v2_number = triangle(mod(k, 3) + 1);

        endOne = rainfallPoints(v1_number, :);
        endTwo = rainfallPoints(v2_number, :);

        if ismember([v1_number v2_number], checkedLines, 'rows') == 0 && ismember([v2_number v1_number], checkedLines, 'rows') == 0
          checkedLines = [checkedLines; [v1_number v2_number] ];
          line2 = [endOne(1) endOne(2) endTwo(1) endTwo(2)];
          [intersect_x, intersect_y] = lineintersect(line, line2);
          if ~isnan(intersect_x)
            h=plot(intersect_x,intersect_y,'rx','MarkerSize',10,'LineWidth',2);
            intersect_z = findValueFromSurface(intersect_x, intersect_y, triangleVectors(j,:), rainfallPoints(triangle(1),:));
            interpolationPoints = [interpolationPoints; [intersect_x, intersect_y, intersect_z]];
          end
        end
      end
    end
    clear('checkedLines');
  end
end;
\end{lstlisting}



\section{Wyznaczanie objętości opadu}

\begin{lstlisting}
pointsInsideCatchment = stationsInsideCatchment(catchment(:,1), catchment(:,2), rainfallPoints);    
triPoints = [borderPoints; interpolationPoints; pointsInsideCatchment];

boarderPtsAmmount = length(borderPoints);
    
C = [(1:(boarderPtsAmmount - 1))' (2:boarderPtsAmmount)'; boarderPtsAmmount 1];
triangulation = delaunayTriangulation(triPoints(:, 1), triPoints(:, 2), C);

disp('Wyznaczanie trójkątów wewnątrz granicy zlewni.');
IO = isInterior(triangulation);
properTriangles = triangulation(IO, :);

disp('Wyznaczanie objętości opadu w poszczególnych trójkątach i całej zlewni.');

properTrianglesAmmount = length(properTriangles);
volumes = zeros(properTrianglesAmmount, 2);
for i = 1:properTrianglesAmmount
  triangle = properTriangles(i, :);
        
  points = [
    triPoints(triangle(1),:);
    triPoints(triangle(2),:);
    triPoints(triangle(3),:);
  ];

  baseField = polyarea([points(:,1);points(1,1)], [points(:,2);points(1,2)]); % m2
  avgHeight = mean(points(:,3)) * 0.001;  % 1 mm = 0.001 m
  volumes(i, 1) = baseField * avgHeight;     % w m3
  volumes(i, 2) = precipOnTriangle(@paraboloidalPrecip, points(:,1)', points(:,2)') / 1000;
  %  volumes(i, 2) = precipOnTriangle(@rationalPrecip, points(:,1)', points(:,2)') / 1000; % precipOnTriangle() daje wynik [mm * m * m] => [mm * m * m]/1000 = m^3
end

disp('Objętości opadu w poszczególnych trójkątach zapisana w macierzy volumes.');
totalRainfall = sum(volumes(:,1));
totalRealRainfall = sum(volumes(:,2));
diff = (totalRealRainfall - totalRainfall) / totalRealRainfall * 100;
    
disp(strcat('Łączny opad powierzchniowy dla zlewni wynosi', {' '}, num2str(totalRainfall),{' '}, 'm3'));
disp(strcat('Rzeczywista objętość opadu dla zlewni wynosi', {' '}, num2str(totalRealRainfall),{' '}, 'm3'));
disp(strcat('Różnica wynosi', {' '}, num2str(diff),{' '}, '%'));
\end{lstlisting}
