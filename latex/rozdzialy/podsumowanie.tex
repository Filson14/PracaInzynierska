\chapter{Podsumowanie}

Celem pracy była analiza danych opadowych zamiarem oszacowania ilości wody jaka spadła na zadany obszar, co przeprowadzane odpowiednio wcześnie mogłoby być sygnałem dla rozpoczęcia procesu reagowania przeciwpowodziowego. Przeprowadzono przegląd istniejących metod wyznaczania opadu powierzchniowego na podstawie pomiarów punktowych, a także możliwości dostępu do takich pomiarów. Znaleziono oraz przetestowano metodę, która pozwoliła uzyskać zadowalające rezultaty bazując na danych symulowanych.

\section{Możliwe rozszerzenia}
Oczywiście zastosowane rozwiązanie można poddawać rozszerzeniu. Przede wszystkim należy wspomnieć o~zastosowaniu rzeczywistych pomiarów z~istniejących posterunków opadowych. Wówczas rezultaty mogłyby być bardziej miarodajne. Pamiętać jednak należy o~nawet kilkumiesięcznym okresie oczekiwania na udzielenie do nich dostępu.

Odwołując się do prawdziwych pomiarów, niniejszy projekt można rozszerzyć o~uwzględnianie ukształtowania terenu wyznaczając tym sposobem kierunek lub nawet prędkość przesuwania się mas wodnych w~kontekście ochrony przeciwpowodziowej lub systemów alarmowania.

Innym usprawnieniem może być użycie bardziej zaawansowanej metody konwersji współrzędnych geograficznych. Zaproponować można chociażby wyznaczanie współczynnika konwersji (metrycznej długości jednego stopnia długości geograficznej na zadanej szerokości geograficznej) osobno dla każdego z~analizowanych punktów. W przypadku analiz rzeczywistych posterunków zmiana ta może być wielce istotna, nie powodująca zakłamań w~powierzchniach zadanego terenu.