\documentclass[11pt]{aghdpl}
% \documentclass[en,11pt]{aghdpl}  % praca w języku angielskim

% Lista wszystkich języków stanowiących języki pozycji bibliograficznych użytych w pracy.
% (Zgodnie z zasadami tworzenia bibliografii każda pozycja powinna zostać utworzona zgodnie z zasadami języka, w którym dana publikacja została napisana.)
\usepackage[english,polish]{babel}

% Użyj polskiego łamania wyrazów (zamiast domyślnego angielskiego).
\usepackage{polski}

\usepackage[utf8]{inputenc}

% dodatkowe pakiety
\usepackage{graphicx}
\DeclareGraphicsExtensions{.png, .jpg}
\graphicspath{ {./img/} }

\usepackage{mathtools}
\usepackage{amsfonts}
\usepackage{amsmath}
\usepackage{amsthm}
\usepackage{longtable}

\usepackage{listings}
\usepackage{color}

\definecolor{purple}{rgb}{0.7,0.1,0.40}

\lstset{
	frame=tb,
	aboveskip=3mm,
	belowskip=3mm,	
	language=Matlab,%
    %basicstyle=\color{red},
    breaklines=true,%
    morekeywords={matlab2tikz},
    keywordstyle=\color{blue},%
    morekeywords=[2]{1}, keywordstyle=[2]{\color{black}},
    identifierstyle=\color{black},%
    stringstyle=\color{purple},
    commentstyle=\color{green},%
    showstringspaces=false,%without this there will be a symbol in the places where there is a space
    numbers=left,%
    numberstyle={\tiny \color{black}},% size of the numbers
    numbersep=9pt % this defines how far the numbers are from the text
}
% --- < bibliografia > ---

%\usepackage[
%style=numeric,
%sorting=none,
%%
%% Zastosuj styl wpisu bibliograficznego właściwy językowi publikacji.
%language=autobib,
%autolang=other,
%% Zapisuj datę dostępu do strony WWW w formacie RRRR-MM-DD.
%urldate=iso8601,
%% Nie dodawaj numerów stron, na których występuje cytowanie.
%backref=false,
%% Podawaj ISBN.
%isbn=true,
%% Nie podawaj URL-i, o ile nie jest to konieczne.
%url=false,
%%
%% Ustawienia związane z polskimi normami dla bibliografii.
%maxbibnames=3,
%]{biblatex}
%\addbibresource{bibliografia.bib}
%
%\usepackage{csquotes}
%% Ponieważ `csquotes` nie posiada polskiego stylu, można skorzystać z mocno zbliżonego stylu chorwackiego.
%\DeclareQuoteAlias{croatian}{polish}
%
%
%
%% Nie wyświetlaj wybranych pól.
%%\AtEveryBibitem{\clearfield{note}}
%

% ------------------------
% --- < listingi > ---

% Użyj czcionki kroju Courier.
\usepackage{courier}

\usepackage{listings}
\lstloadlanguages{TeX}

\lstset{
	literate={ą}{{\k{a}}}1
           {ć}{{\'c}}1
           {ę}{{\k{e}}}1
           {ó}{{\'o}}1
           {ń}{{\'n}}1
           {ł}{{\l{}}}1
           {ś}{{\'s}}1
           {ź}{{\'z}}1
           {ż}{{\.z}}1
           {Ą}{{\k{A}}}1
           {Ć}{{\'C}}1
           {Ę}{{\k{E}}}1
           {Ó}{{\'O}}1
           {Ń}{{\'N}}1
           {Ł}{{\L{}}}1
           {Ś}{{\'S}}1
           {Ź}{{\'Z}}1
           {Ż}{{\.Z}}1,
	basicstyle=\footnotesize\ttfamily,
}

% ------------------------

\AtBeginDocument{
	\renewcommand{\tablename}{Tabela}
	\renewcommand{\figurename}{Rys.}
}

% ------------------------
% --- < tabele > ---

\usepackage{array}
\usepackage{tabularx}
\usepackage{multirow}
\usepackage{booktabs}
\usepackage{makecell}
\usepackage[flushleft]{threeparttable}
\usepackage{siunitx}

% defines the X column to use m (\parbox[c]) instead of p (`parbox[t]`)
\newcolumntype{C}[1]{>{\hsize=#1\hsize\centering\arraybackslash}X}


%---------------------------------------------------------------------------

\author{Filip Pasternak}
\shortauthor{F. Pasternak}

\titlePL{Akwizycja danych dla identyfikacji prostego modelu opad-odpływ.}
\titleEN{Data acquisition for identification of a simple rainfall runoff model.}
\title{Praca inżynierska}

\shorttitlePL{Akwizycja danych dla identyfikacji prostego modelu opad-odpływ.} % skrócona wersja tytułu jeśli jest bardzo długi
\shorttitleEN{Data acquisition for identification of a simple rainfall runoff model.}

\thesistype{Praca dyplomowa inżynierska}

\supervisor{dr inż. Janusz Miller}
%\supervisor{Janusz Miller PhD}

\degreeprogramme{Informatyka}
%\degreeprogramme{Computer Science}

\date{2015}

\department{Katedra Informatyki Stosowanej}
%\department{Department of Applied Computer Science}

\faculty{Wydział Elektrotechniki, Automatyki,\protect\\[-1mm] Informatyki i Inżynierii Biomedycznej}
%\faculty{Faculty of Electrical Engineering, Automatics, Computer Science and Biomedical Engineering}

\acknowledgements{Work hard and never give up!
Bardzo dziękuję mojemu promotorowi za okazaną pomoc przy tworzeniu tej pracy.}


\setlength{\cftsecnumwidth}{10mm}

%---------------------------------------------------------------------------
\setcounter{secnumdepth}{4}

\begin{document}

\titlepages

% Ponowne zdefiniowanie stylu `plain`, aby usunąć numer strony z pierwszej strony spisu treści i poszczególnych rozdziałów.
\fancypagestyle{plain}
{
	% Usuń nagłówek i stopkę
	\fancyhf{}
	% Usuń linie.
	\renewcommand{\headrulewidth}{0pt}
	\renewcommand{\footrulewidth}{0pt}
}

\tableofcontents
\clearpage

\newpage
\chapter{Wstęp}
\section{Przedmowa}
Natura to nieposkromiona siła, którą człowiek stara się poznać. Choć już od lat 60-tych prowadzone są badania nad systemami modyfikacji pogody (w szczególności do zastosowań militarnych), to prawidłowe przewidywanie zjawisk atmosferycznych, a także ich skutków, wciąż stanowi problem wymagający rozwiązania.

W lecie 2010 roku Polskę zaatakowała ogromna powódź, która dotknęła niemal wszystkich województw. W wyniku wzmożonych opadów rzeki wezbrały i w wielu miejscach wystąpiły z koryta. Mimo istnienia około stu zbiorników retencyjnych oraz wielu zapór wodnych, nie udało się zapobiec tragedii, w~wyniku której mnóstwo ludzi straciło swój dobytek.

Istniejąca obecnie infrastruktura dokonująca pomiarów zjawisk pogodowych czy stanów zbiorników wodnych może stanowić bazę do systemów umożliwiających zapobieganie takim tragediom. Analiza wielkości opadu oraz przewidywanie drogi jej spływu może pozwolić na przykład operatorom zapór wodnych na podjęcie właściwych, a~przede wszystkim odpowiednio wczesnych działań tak, aby przygotować zbiorniki retencyjne na przyjęcie dodatkowej ilości wody oraz w~sposób kontrolowany pokierować jej spływ i zabezpieczyć ludność.

Praca ta traktować będzie o~elemencie składowym przybliżonego powyżej rozwiązania, czyli analizie danych opadowych.

\section{Opis dokumentu}
% Sekcja do rozbudowy w miarę rozszerzenia dokumentu.
Praca ta porusza tematykę analizy danych opadowych pochodzących z~posterunków opadowych. Opisuje teorię przekształcenia wartości punktowej opadu na wartość powierzchniową w~obszarze wskazanej zlewni rzeki.

\textbf{Odczyt danych} rozdział ten opisuje proces przeprowadzonej analizy serwisu \cite w~celu dostępu do danych meteorologicznych i hydrologicznych.

\textbf{Interpolacja danych punktowych opadów} w tej części opisano algorytm triangulacji metodą Delaunaya oraz sposób przekształcenia wartości punktowych opadów na obszar zlewni.
\section{Cele pracy}
Celem pracy jest analiza możliwości uzyskania danych pogodowych, zbudowanie mechanizmu przekształcającego wskazania z~posterunków opadowych na objętość opadu w~zadanym obszarze, a~także przeprowadzenie badania wpływu usunięcia poszczególnych posterunków na wartości wynikowe.

%Celem pracy jest analiza korelacji danych opadowych i~odpływowych na rzece. Obliczenia opierać się będą na informacjach ze strony pogodynka.pl udostępnianej przez Instytut Meteorologii i Gospodarki Wodnej. Wszelkie działania i obliczenia przeprowadzone zostaną w~środowisku Matlab (darmowa wersja próbna R2015b). Elementy składowe pracy to:
\begin{itemize}
\item
Odczyt z serwisu pogodynka.pl danych posterunków opadowych i wodowskazowych (takich jak nazwa i lokalizacja), wartości odpowiednich wskazań posterunków oraz parametrów granic zlewni.
\item
Przekształcenie danych opadowych metodą triangulacji Delaunaya i aplikacja tej metody do przebiegu granic zlewni.
\item
Analiza korelacji opadu powierzchniowego z~wysokością poziomu wody na odpowiednich posterunkach wodowskazowych.
\end{itemize}
\chapter{Odczyt danych pogodowych}
W celu odczytu potrzebnych danych napisano funkcje w~środowisku Matlab. Opierają się na zapytaniu HTTP do API dostępnego na stronie monitor.pogodynka.pl i właściwym parsowaniu odpowiedzi.

Dane z serwera odbierane są w formacie JSON. Do jego obsługi w~Matlab'ie zastosowano bibliotekę JSONlab, która zawiera funkcję loadjson(). Samo zapytanie wykonywane jest poprzez funkcję webread().

Funkcja loadjson() przyjmuje łańcuch znaków zawierający opis obiektu JSON. W wyniku jej działania każdy taki obiekt przekształcany jest do struktury danych Matlab zwanej \textbf{cell array} zawierającej indeksowane kontenery mogące przechowywać różne typy danych. Tablice JSON są przekształcane do tablic Matlab. 

Kody źródłowe poszczególnych funkcji znajdują się w~podrozdziale~\ref{sec:kod_odczyt_danych}.
\section{Dane posterunków}
Za pomocą odpowiednich funkcji do pamięci zmiennych środowiska ładowane są informacje na temat posterunków opadowych i wodowskazowych. Zapisywane są tutaj ich współrzędne geograficzne, nazwa oraz identyfikator, który będzie użyty do pobierania pomiarów dla danego posterunku.

\section{Dane zlewni}
Pobierane dane zlewni obejmują nazwy oraz współrzędne geograficzne punktów stanowiących ich granice i są zapisywane w kontenerze cell array. Dla ułatwienia przeprowadzenia analizy dla pojedynczej zlewni zastosowano wydzielenie poszczególnych elementów kontenera do osobnej zmiennej.

\section{Pomiar opadu}
\label{sec:pomiar_opadu}
Została napisana funkcja, która dla zadanych na wejściu posterunków opadowych odczytuje pomiary ze wskazanego na początku rozdziału serwisu. Funkcja ta zwraca strukturę zawierającą poszczególne pomiary.

Pojedynczy element zawiera dane ostatniego pomiaru godzinowego i~dobowego oraz dwie tablice z~godzinowymi pomiarami za ostatnie 48 godzin i dobowymi na przestrzeni tygodnia.

Odczytane dane zostają zapisane automatycznie w~pliku .mat.

\section{Pomiar poziomu wody}
Funkcja odczytująca wskazania posterunków wodowskazowych działa analogicznie jak ta, opisana w podrozdziale~\ref{sec:pomiar_opadu}. Pojedynczy element kontenera wynikowego zawiera aktualne wskazanie poziomu wody (wraz z datą pomiaru), tablicę godzinowych pomiarów poziomu wody za ostatnie 72 godziny oraz tablicę godzinowych pomiarów wielkości przepływu (w jednostce $m^3/s$) za taki sam okres czasu.
\chapter{Podstawy teoretyczne}
Jak wspomniano w rozdziale~\ref{sec:cele}, dane wejściowe dla programu są wskazaniami pomiarów z~poszczególnych posterunków opadowych. Są to dane punktowe, zatem aby oszacować ilość wody jaka spadła na zadanym obszarze konieczne jest przeprowadzenie ich przekształcenia w~celu wyznaczenia opadu powierzchniowego.

\section{Powszechne metody wyznaczania opadu powierzchniowego}

Istnieje wiele metod wyznaczania średniego opadu powierzchniowego. Począwszy od prostego obliczenia średniej arytmetycznej wartości z~posterunków opadowych, poprzez metody wieloboków, izohiet po hipsometryczną. Różnią się one dokładnością, sposobem aproksymacji czy też poziomem zaawansowania. Każda jest też dostosowana do odpowiednich warunków analizowanego terenu. Poniżej znajduje się ich krótkie przybliżenie~\cite{obliczanie_opadu_sredniego, metody_obliczania_pb, opad_metody}.

\subsection{Metoda Thiessena}
Zwana także metodą wieloboków lub wielokątów równego zadeszczenia. Preferowana w~przypadku terenów nizinnych o~niewielkim zróżnicowaniu rzeźby. Walorem dla jej zastosowania jest równomierne rozmieszczenie posterunków opadowych. Stosuje się tu triangulację, a~następnie wyznaczając symetralne boków powstałych trójkątów tworzy się wieloboki, które odpowiadają poszczególnym posterunkom opadowym. Wskazanie posterunku przypisywane jest do wieloboku i~mnożone przez jego powierzchnię (uwzględniając granice zlewni). Sumując te iloczyny otrzymuje się oczekiwane wskazanie opadu powierzchniowego.

\begin{figure}[!ht]
	\begin{subfigure}{.5\textwidth}
		\centering
		\includegraphics[width=0.7\linewidth]{thiessen1}
		\caption{Triangulacja i~symetralne boków.}
	\end{subfigure}%
	\begin{subfigure}{.5\textwidth}
		\centering
		\includegraphics[width=0.7\linewidth]{thiessen2}
		\caption{Wieloboki wyznaczone na obszarze zlewni.}
	\end{subfigure}	
\caption{Schemat dla metody Thiessena (źródło:~\cite{obliczanie_opadu_sredniego}). }
\end{figure}

\subsection{Metoda izohiet}
Metoda ta jest polecana dla analizy obszarów górskich, albowiem uwzględnia zależność między wysokością nad poziomem morza, a~wysokością opadu. Uznawana jest również za najbardziej dokładną.

Polega na podziale obszaru zlewni na fragmenty ograniczane przez kolejne izohiety. Dla każdego z~nich przyjmuje się wartość opadu będącą średnią arytmetyczną opadu na wyznaczających dany fragment izohietach. Taką samą zasadę stosuje się gdy granica zlewni przebiega blisko kolejnej izohiety. Jeżeli odległość ta jest znaczna, przyjmuje się wysokość opadu bliższej izohiety. Suma iloczynów wartości opadu i~powierzchni poszczególnych fragmentów stanowi wartość opadu na zadanej powierzchni.


\subsection{Metoda krzywej hipsometrycznej}
Podobnie jak metoda izohiet, zalecana jest dla badania terenów górskich, a~przede wszystkim małych zlewni. Jest stosunkowo pracochłonna, aczkolwiek dokładna.

W IV ćwiartce układu współrzędnych kreślona jest krzywa hipsometryczna oparta na mapie wysokościowej zlewni. Obrazuje ona zależność wysokości nad poziom morza (oś $X$) i~powierzchni zlewni (oś $Y$). Linia w~ćwiartce II nazywana jest krzywą gradientową opadów atmosferycznych. Jest to korelacja wzniesienia posterunku opadowego na osi odciętych oraz wysokości zmierzonego opadu na osi rzędnych.

Dokonując rzutowania krzywej hipsometrycznej poprzez ćwiartkę III i II (w~oparciu o~krzywą gradientową) na ćwiartkę I wyznaczana jest kolejna krzywa (zwana pluwiometryczną). Pole obszaru znajdującego się pod nią to rozmiar opadu na analizowanym obszarze. Opisany proces obrazuje rysunek~\ref{fig:hipsometryczna}.

\begin{figure}[!ht]
\centering
\includegraphics[width=0.7\linewidth]{hipsometryczna}
\caption{Wykres metody hipsometrycznej (źródło:~\cite{obliczanie_opadu_sredniego}.}
\label{fig:hipsometryczna}
\end{figure}

\subsection{Metoda regionów opadowych}
Analizując zlewnię pod kątem cech fizyczno-geograficznych wydziela się obszary o~podobnych warunkach pogodowych. Wskazanie opadu na każdym z~nich stanowi średnia arytmetyczna pomiarów z~posterunków opadowych znajdujących się wewnątrz niego. Uwzględniając ograniczenie granicami zlewni wylicza się powierzchnię poszczególnych fragmentów i~mnoży ją przez wskazanie opadu. Podobnie jak w~innych metodach - suma takich iloczynów stanowi wielkość opadu w~zlewni.

Metoda ta jest mniej dokładna. Naturalnie, spisuje się najlepiej gdy obszar zlewni jak najbardziej pokrywa się z~wyznaczonymi regionami. Jej atutem jest niewielki poziom skomplikowania.

\begin{figure}[!ht]
\centering
\includegraphics[width=0.5\linewidth]{regiony_opadowe}
\caption{Schemat regionów opadowych (źródło:~\cite{metody_obliczania_pb}).}
\label{fig:regiony_opadowe}
\end{figure}


\section{Zastosowana metoda}
\label{sec:zastosowana_metoda}
Jak wspomniano we wstępie, praca ta traktuje o analizie danych pojedynczego opadu, a~nie zbieranych przez pewien okres czasu. Przy wyborze sposobu przekształcenia danych jest to bardzo istotne kryterium, które wraz z~możliwością jak najlepszego dopasowania do kształtu zlewni stanowiło główne wymagania tej części pracy. Brano także pod uwagę poziom skomplikowania implementacji stosowanego rozwiązania. Wybór padł na połączenie techniki triangulacji wraz z~interpolacją przy użyciu płaszczyzny.

\begin{figure}[!ht]
	\centering
	\includegraphics[scale=0.7]{algorytm}
	\caption{Schemat krokowy zastosowanej metody}
	\label{fig:algorytm}
\end{figure}

%Triangulacja polega na stworzeniu siatki trójkątów o~wierzchołkach w~zadanych punktach (w tym wypadku są to posterunki opadowe o znanej wysokości opadu). 
Zastosowano algorytm triangulacji Delaunay'a, który wprowadza dodatkowe ograniczenie na tworzone trójkąty. Mianowicie, okrąg opisany na każdym z~nich nie może zawierać innych punktów siatki poza wierzchołkami danego trójkąta. Ta metoda ma na celu maksymalizację równoboczności powstałych trójkątów, a~co za tym idzie, równomierność budowanej siatki.

\begin{figure}[!ht]
	\centering
	\includegraphics[scale=0.3]{delaunay}
	\caption{Schemat realizacji warunku Delaunay'a (źródło: \textit{www.wikipedia.pl}).}
	\label{fig:delaunay}
\end{figure}

Po zastosowaniu triangulacji, następnym etapem jest wyznaczenie punktów, dla których należy interpolować wartość opadu. Tymi punktami są poszczególne węzły granicy wskazanej zlewni oraz miejsca przecięcia tej granicy z~krawędziami trójkątów. Wraz z~posterunkami opadowymi znajdującymi się wewnątrz obszaru dla jakiego wyznaczany jest opad powierzchniowy będą stanowiły węzły kolejnej triangulacji.

Wyznaczenie wartości dla wskazanych punktów odbywa się poprzez interpolację płaszczyzną przechodzącą przez trzy punkty (będące wierzchołkami kolejnych trójkątów). Ustalane są współrzędne wektorów normalnych dla poszczególnych płaszczyzn (czyli wektorów prostopadłych do powierzchni danej płaszczyzny), a~następnie korzystając ze wzoru~\ref{eq:wartosc_interpolowana}, wyznaczana jest wartość opadu ($z$) w~punkcie o~zadanych współrzędnych~$x$~i~$y$.

Równanie płaszczyzny przedstawia się następująco.

\begin{equation}
\begin{gathered}
A(x - x_0) + B(y - y_0) + C(z - z_0) = 0 \\
[A, B, C] = (P_2 - P_1) \times (P_3 - P_2)
\label{eq:rownanie_plaszczyzny}
\end{gathered}
\end{equation}
gdzie
% eqwhere z aghdpl powoduje błąd
\begin{description}[leftmargin=3cm, itemsep=0cm, labelsep=0cm]
	\item[$x_0, y_0, z_0$] współrzędne punktu należącego do płaszczyzny,
	\item[{[}$A, B, C${]}] wektor normalny płaszczyzny, %rozwiązać problem z [ ] jako wektor
	\item[$A, B, C$] nie mogą być jednocześnie równe 0.
\end{description}
%
Co po przekształceniu daje
\begin{equation}
\label{eq:wartosc_interpolowana}
	z = \frac{A(x - x_0) + B(y - y_0)}{-C} + z_0
\end{equation}
gdzie $C \neq 0$.


Jeżeli $C$, ze wzoru~\ref{eq:wartosc_interpolowana}, przyjmuje wartość 0 oznacza to, iż w~każdym wierzchołku trójkąta opad był zerowy. Wówczas punktowi interpolowanemu przypisuje się takowe wskazanie.

Po przeprowadzeniu interpolacji wartości dokonuje się ponownej triangulacja, tym razem z~użyciem punktów interpolowanych oraz posterunków wewnątrz zlewni. Nowopowstała siatka trójkątów, wraz z~trzecim wymiarem jakim jest wysokość opadu, tworzy zbiór brył o~trójkątnych podstawach. Wyznaczając i~sumując objętość opadu w~każdej z~nich uzyskiwany jest łączny opad na wskazanej zlewni~\cite{matematyka_poradnik, mathMonthly}.

\begin{equation}
	V = \sum_{i=1}^{n}P_{pi} \cdot \frac{h_{i1}+h_{i2}+h_{i3}}{3}
\label{eq:opad_powierzchniowy}
\end{equation}
gdzie
\begin{description}[leftmargin=3cm, itemsep=0cm, labelsep=0cm]
	\item[$V$] łączna objętość opadu,
	\item[$n$] ilość wyznaczonych brył,
	\item[$P_{pi}$] pole podstawy $i$-tej bryły,
	\item[$h_{i1}, h_{i2}, h_{i3}$] długość krawędzi bocznej $i$-tej bryły (wartość opadu) w~danym wierzchołku.
\end{description}




\section{Możliwe rozszerzenia algorytmu}
Oczywiście można dążyć do usprawnienia metody przekształcania danych wejściowych.
Przykładowe rozszerzenia mogą obejmować
\begin{itemize}
\item{ Analizę sąsiednich dla trójkąta punktów i~na ich podstawie wyznaczenie wartości opadu w~środku zadanego fragmentu, a~dalej dzielenie go na mniejsze. }

\item{ Interpolację powierzchnią inną niż płaszczyzna. Kształt powierzchni definiować na bazie większej ilości punktów znanych. }

\item{ Uwzględnienie ukształtowania terenu podczas analizy obszaru zlewni. }
\end{itemize}
\chapter{Implementacja}
\label{cha:implementacja}
%Możliwości wyboru technologii dla rozwiązania problemu omawianego w~niniejszej pracy jest bardzo wiele. Począwszy od popularnych języków obiektowych jak Java, która swoimi, niemal nieograniczonymi możliwościami i~ogromem rozszerzających je bibliotek, pozwala rozwiązywać dowolne problemy czy technologii webowych jak JavaScript, która  

Praktyczna część niniejszej pracy została zrealizowana w~środowisku Matlab 2015b (darmowa wersja próbna). Łączy on niemal wszystkie kwestie poruszane w~niniejszej pracy - zaawansowane obliczenia matematyczne, przetwarzanie zbioru danych, pozwala na dynamiczne budowanie wykresów do prezentacji. Sprawdza się idealnie w~przypadku zadań analitycznych bądź symulacyjnych, a~jego konfiguracja sprowadza się do instalacji samego programu. Triangulację przeprowadzono w~oparciu o~klasę \textit{delaunayTriangulation}.

\section{Konwersja współrzędnych geograficznych}
Współrzędne poszczególnych punktów wejściowych określane są poprzez współrzędne geograficzne w~formacie dziesiętnym. Wobec tego, konieczne jest przekształcenie ich na wartości metryczne, aby móc wyznaczyć wartość opadu w~jednostkach SI.

Jednostki wzdłuż osi $Y$ przekształcone zostały stosując zależność $1^o=111196.672 [m]$, ponieważ w~przypadku szerokości geograficznej jest ona stała.

Dla długości geograficznej sytuacja nie jest tak prosta, gdyż długość jednego stopnia zależy od równoleżnika, na którym dokonuje się pomiaru. Przyjęto, że konwersja zostanie przeprowadzona w~oparciu o~długość równoleżnika leżącego na średniej szerokości geograficznej danych wejściowych. Praca zorientowana jest na analizę opadu z~konkretnej zlewni, zatem różnica z~wartością rzeczywistą jest niewielka (zostanie podana dla konkretnego przypadku). Przekształcenia dokonano w~oparciu o~wzór~\ref{eq:konwersja_wspolrzednych}. Do obliczeń przyjęto promień równikowy Ziemi o~wartości 6378410~m.

\begin{equation}
	r = \cos(\alpha)*R \Rightarrow 1^o = \frac{2 \pi r}{360} [m] \\
	\label{eq:konwersja_wspolrzednych}
\end{equation}
gdzie
\begin{description}[leftmargin=2cm, itemsep=0cm, labelsep=0cm]
	\item[$\alpha$] kąt, będący wartością szerokości geograficznej
	\item[$R$] równikowy promień Ziemi
	\item[$r$] promień równoleżnika na szerokości geograficznej $\alpha$
\end{description}

\section{Wielkość opadu}
Z~powodu braku dostępu do rzeczywistych danych meteorologicznych oraz hydrologicznych, co zostało opisane w~rozdziale~\ref{cha:odczyt_danych}, wykluczona została możliwość racjonalnej i~miarodajnej analizy pomiarów wraz z~określeniem korelacji opadu i~wysokości poziomu wody rzeki. Wobec powyższego, zdecydowano o~zastosowaniu danych symulowanych. Stworzone zostały funkcje generujące pomiar opadu we wskazanym punkcie. Przygotowane zostały dwa warianty - opad elipsoidalny oparty na wzorze~\ref{eq:opad_elipsa} oraz zdefiniowany funkcją wymierną~\ref{eq:opad_wymierny}.

\begin{equation}
A = 1%z = h - \alpha * (x - x_0)^2 - \betha * (y - y_0)^2;
\label{eq:opad_elipsa}
\end{equation}

\begin{equation}
z = MAX\_PRECIP * \frac{-1}{h+(x-x_0)^2 + \alpha * (y-y_0)^2}
\label{eq:opad_wymierny}
\end{equation}

Zdefiniowano również funkcję, która ma za zadanie wyznaczenie objętości rzeczywistego opadu we wskazanym trójkącie w~oparciu o~zadany wariant opadu. Tym sposobem zostało rozwiązane zagadnienie weryfikacji dokładności działania algorytmu opisanego w~podrozdziale~\ref{sec:zastosowana_metoda}. Uzyskiwane wyniki porównywane są z~danymi pseudorzeczywistymi.

\begin{figure}[ht]
\centering
	\begin{subfigure}{.5\textwidth}
		\centering
		\includegraphics[width=1\linewidth]{chmura_elipsoidalna_1}
		\caption{Wykres zastosowanej funkcji elipsoidalnej}
		\label{fig:elipsa_3d}
	\end{subfigure}%	
	\begin{subfigure}{0.5\textwidth}
		\centering
		\includegraphics[width=1\linewidth]{chmura_elipsoidalna_2}
		\caption{Rzut wykresu zastosowanej funkcji elipsoidalnej}
		\label{fig:elipsa_2d}
	\end{subfigure}	
	
	\begin{subfigure}{.5\textwidth}
		\centering
		\includegraphics[width=1\linewidth]{chmura_wymierna_1}
		\caption{Wykres zastosowanej funkcji wymiernej}
		\label{fig:elipsa_3d}
	\end{subfigure}%	
	\begin{subfigure}{0.5\textwidth}
		\centering
		\includegraphics[width=1\linewidth]{chmura_wymierna_2}
		\caption{Rzut wykresu zastosowanej funkcji wymiernej}
		\label{fig:elipsa_2d}
	\end{subfigure}	
\caption{Funkcje wartości opadu}
\end{figure}

\section{Zastosowana metoda}

Jak wspomniano na początku tego rozdziału, przeprowadzenie triangulacji oparto o~klasę delaunayTriangulation. Powstała siatka trójkątów została zaprezentowana na rysunku \ref{fig:triagnulacja_danych}.

\begin{figure}[ht]
	\centering
\begin{subfigure}[t]{.5\textwidth}
	\centering
	\includegraphics[width=0.7\linewidth]{dane_wejsciowe}
	\caption{Posterunki opadowe i~obszar zlewni}
	\label{fig:dane_wejsciowe}
\end{subfigure}%	
	\begin{subfigure}[t]{0.5\textwidth}
	\centering
	\includegraphics[width=0.7\linewidth]{triangulacja_zlewnia}
	\caption{Siatka triangulacji dla posterunków}
	\label{fig:triagnulacja_danych}
\end{subfigure}	
\caption{Prezentacja danych wejściowych}
	
\end{figure}


\begin{figure}[ht]
	\centering
	\includegraphics[scale=0.5]{punkty_do_interpolacji}
	\caption{Punkty wyznaczone do interpolacji.
	O - węzły granicy, X - przecięcia granicy z~liniami siatki.}
	\label{fig:punkty_interpolacji}
\end{figure}

Wyznaczenie wartości dla powstałych punktów odbywa się poprzez interpolację powierzchniową poszczególnych trójkątów. Została stworzona funkcja wyznaczająca wektor normalny dla pojedynczego trójkąta. Z~jego pomocą (oraz znając współrzędne punktu należącego do płaszczyzny - np. jeden z~wierzchołków trójkąta) wyznacza się wartość \textit{z} punktu o~współrzędnych $x$~i~$y$~w~oparciu o~równanie ogólne płaszczyzny.

Przy użyciu przygotowanej funkcji (opartej na wzorze~\ref{eq:wartosc_interpolowana}, dla wszystkich wyznaczonych do interpolacji punktów uzyskiwana jest wartość opadu korzystając ze współrzędnych $x$~i~$y$ punktu i~wektora normalnego trójkąta, w~którym ten punkt się znajduje.

\begin{figure}[ht]
	\centering
	\includegraphics[scale=0.5]{prezentacja_interpolacji}
	\caption{Interpolacja płaszczyzną dla pojedynczego trójkąta}
	\label{fig:plaszczyzna_interpolacji}
\end{figure}


Wyizolowane punkty, krytyczne z~punktu widzenia zadanego obszaru opadu, stanowią węzły kolejnej triangulacji. Jak zaprezentowano na rysunku~\ref{fig:druga_triangulacja}, pojawia się komplikacja wynikająca z~łączenia ze sobą wszystkich pobliskich punktów. Tym sposobem kreowane są trójkąty, które nie wprowadzają informacji dla rozwiązania. Co więcej, wpływają negatywnie na ewentualny wynik.

\begin{figure}[ht]
	\centering
\begin{subfigure}[t]{1\textwidth}
	\centering
	\includegraphics[width=1\linewidth]{dane_druga_triangulacja}
	\caption{Punkty dla ponownej triangulacji}
	\label{fig:dane_druga_triangulacja}
\end{subfigure}	
	\begin{subfigure}[t]{1\textwidth}
	\centering
	\includegraphics[width=1\linewidth]{druga_triangulacja}
	\caption{Siatka triangulacji dla wyznaczonych punktów}
	\label{fig:druga_triagnulacja}
\end{subfigure}	
\caption{Triangulacja danych interpolowanych}
\end{figure}

Problem ten rozwiązano poprzez dodanie ograniczenia dla obiektu klasy \textit{delaunayTriangulation} zdefiniowanego jako zbiór węzłów triangulacji stanowiących granicę zlewni poddawanej analizie. Odfiltrowanie nieistotnych fragmentów wykonano stosując metodę \textit{isInterior}, która wskazuje czy poszczególne trójkąty utworzonej siatki zawierają się w~obszarze wskazanym przez ograniczenie. Rezultat tych kroków prezentuje rysunek~\ref{fig:druga_triangulacja_ograniczenie}

\begin{figure}[ht]
	\centering
	\includegraphics[scale=0.5]{druga_triangulacja_z_ograniczeniem}
	\caption{Triangulacja z~ograniczeniem obszarem zlewni}
	\label{fig:druga_triangulacja_ograniczenie}
\end{figure}
\chapter{Opracowanie wyników}

Rozdział ten zawiera analizę wyników działania programu opisanego w~części~\ref{cha:implementacja}. Przeprowadzono także weryfikację wpływu usunięcia pojedynczego posterunku opadowego na uzyskane wskazanie opadowe na obszarze, co było jednym z~celów tej pracy.


\section{Wielkość błędu}
asdasd

\section{Warianty opadu}
Opisany w~rozdziale~\ref{cha:implementacja} program został uruchomiony celem wyznaczenia wartości opadu powierzchniowego dla obu przygotowanych wariantów funkcji opadowej. Poniżej zaprezentowano uzyskane wyniki.

\begin{table}[!ht]
\caption{Wyniki działania programu}
\begin{center}
\begin{tabular}{|c|c|c|c|}
\hline
 Wariant      & Opad wyznaczony [$m^3$] & Opad rzeczywisty [$m^3$] & Różnica [\%] \\ \hline \hline

 Paraboliczny & 1 462 995 563.4253  &   1 500 568 329.7155  &  -2.5039 \\ \hline
 Wymierny     & 1 559 081 843.8972  &   1 571 689 381.8665  &  -0.8022 \\ \hline

\end{tabular}
\end{center}
\end{table}

Jak można zauważyć, dla wariantu z~opadem parabolicznym mamy do czynienia z~niedomiarem. Jest to spodziewana sytuacja wynikająca z~interpolowania funkcji wklęsłej. Różnica z~rzeczywistą objętością opadu wynosi ok. 2.5 \%, zatem jest to zadowalający wynik.

W przypadku wymiernego opadu uzyskano jeszcze lepsze wyniki. Błąd wyznaczonej wartości jest mniejszy niż jeden procent, a~należy zwrócić uwagę, dane przygotowane na potrzeby testów obejmują obszar rzędu stu tysięcy kilometrów kwadratowych.



\section{Analiza wrażliwości na posterunki opadowe}
W~tej części przeprowadzono wpływ wyłączenia wskazanego posterunku opadowego (ograniczono się do wybory punktów zawartych w granicach wskazanej zlewni) na wynik działania omawianej metody. Pozwala to znaleźć odpowiedź na pytanie: \textit{Czy sieć istniejących posterunków opadowych jest wystarczająca?}, a~to pomoże podejmować decyzje o~zwiększaniu ilości owych posterunków bądź przeniesieniu w~inną lokalizację.

Rysunek~\ref{fig:identyfikatory} przedstawia identyfikatory wykluczanych z~analizy punktów, natomiast wiersze tabeli prezentują rezultat działania programu bez zadanego punktu.

\begin{figure}[!ht]
	\centering
	\includegraphics[width=1\linewidth]{identyfikatory_punktow}
	\label{fig:identyfikatory}
	\caption{Identyfikatory posterunków wewnątrz zlewni}
\end{figure}

\subsection{Opad paraboliczny}

\begin{table}[!ht]
\label{tab:wyniki_wymierna}
\caption{Wpływ usunięcia posterunku na wynik algorytmu. Wariant paraboliczny.}
\begin{center}
\begin{tabular}{|c|l|l|r@{.}l|r@{.}l|r@{.}l|}
%nagłówek tabeli
\cline{2-3} \cline{6-9}
\multicolumn{1}{l}{} & \multicolumn{2}{|c|}{Opad $[m^3]$} & \multicolumn{2}{c}{} & \multicolumn{4}{|c|}{Wpływ posterunku} \\
\hline Posterunek & Wyznaczony & Rzeczywisty & \multicolumn{2}{c}{Różnica [\%]} & \multicolumn{2}{|c|}{[$m^3$]}& \multicolumn{2}{|c|}{$\times 10^{-3} [\%]$} \\ \hline \hline

--    &     1462995563.42529   &    1500568329.7155    &      -2 & 5039  &              0 & 00          &     0 & 00 \\ \hline
1   &     1463076745.52465   &    1500568302.6776    &      -2 & 4984  &          81182 & 0994   &     5 & 55 \\ \hline
2   &     1461776787.89791   &    1500568329.7155    &      -2 & 5851  &       -1218775 & 5274   &   -83 & 31 \\ \hline
3   &     1462473733.66747   &    1500568329.7155    &      -2 & 5386  &        -521829 & 7578   &   -35 & 67 \\ \hline
4   &     1460378875.61033   &    1500568329.7155    &      -2 & 6782  &       -2616687 & 8150   &  -178 & 86 \\ \hline
5   &     1468988504.71699   &    1500568343.3016    &      -2 & 1045  &        5992941 & 2917   &   409 & 64 \\ \hline
6   &     1461622417.1231    &    1500568329.7155    &      -2 & 5954  &       -1373146 & 3022   &   -93 & 86 \\ \hline
7   &     1464682183.90386   &    1500568361.4533    &      -2 & 3915  &        1686620 & 4786   &   115 & 29 \\ \hline
8   &     1462995563.42529   &    1500568329.7155    &      -2 & 5039  &             -0 & 0000   &     0 & 00 \\ \hline
9   &     1453656531.65409   &    1500568329.7155    &      -3 & 1262  &       -9339031 & 7712   &  -638 & 35 \\ \hline
10  &     1462338580.82334   &    1500568329.7155    &      -2 & 5476  &        -656982 & 6020   &   -44 & 91 \\ \hline
11  &     1454316634.11975   &    1500568329.7155    &      -3 & 0822  &       -8678929 & 3055   &  -593 & 23 \\ \hline
12  &     1460880307.4094    &    1500568329.7155    &      -2 & 6448  &       -2115256 & 0159   &  -144 & 58 \\ \hline
13  &     1460206170.95541   &    1500568329.7155    &      -2 & 6897  &       -2789392 & 4699   &  -190 & 66 \\ \hline
14  &     1461450430.25043   &    1500568329.7155    &      -2 & 6068  &       -1545133 & 1749   &  -105 & 61 \\ \hline
15  &     1462002685.70473   &    1500568329.7155    &      -2 & 5700  &        -992877 & 7206   &   -67 & 87 \\ \hline
16  &     1462214551.12403   &    1500568329.7155    &      -2 & 5559  &        -781012 & 3013   &   -53 & 38 \\ \hline
17  &     1462691315.16175   &    1500568329.7155    &      -2 & 5241  &        -304248 & 2635   &   -20 & 80 \\ \hline
18  &     1464582760.98511   &    1500568329.7155    &      -2 & 3981  &        1587197 & 5598   &   108 & 49 \\ \hline
19  &     1461064560.9566    &    1500568329.7155    &      -2 & 6325  &       -1931002 & 4687   &  -131 & 99 \\ \hline
20  &     1461310691.73423   &    1500568329.7155    &      -2 & 6161  &       -1684871 & 6911   &  -115 & 17 \\ \hline
21  &     1459561769.38071   &    1500568329.7155    &      -2 & 7327  &       -3433794 & 0446   &    -0 & 23 \\ \hline
22  &     1463418751.62474   &    1500568331.6960    &      -2 & 4757  &         423188 & 1994   &    -0 & 03 \\ \hline
23  &     1453767799.36506   &    1500568329.7155    &      -3 & 1189  &       -9227764 & 0602   &    -0 & 63 \\ \hline
24  &     1462113590.39904   &    1500568331.1793    &      -2 & 5627  &        -881973 & 0263   &    -0 & 06 \\ \hline
25  &     1461120846.70184   &    1500568329.7155    &      -2 & 6288  &       -1874716 & 7235   &    -0 & 13 \\ \hline
26  &     1462704787.16475   &    1500568329.7155    &      -2 & 5233  &        -290776 & 2605   &    -0 & 02 \\ \hline
27  &     1460220263.75282   &    1500568329.7155    &      -2 & 6889  &       -2775299 & 6725   &    -0 & 19 \\ \hline
28  &     1461578202.47148   &    1500568329.7155    &      -2 & 5984  &       -1417360 & 9538   &    -0 & 10 \\ \hline
29  &     1461321743.5352    &    1500568329.7155    &      -2 & 6154  &       -1673819 & 8901   &    -0 & 11 \\ \hline
30  &     1460813336.266     &    1500568329.7155    &      -2 & 6493  &       -2182227 & 1593   &    -0 & 15 \\ \hline
31  &     1460877238.93555   &    1500568329.7155    &      -2 & 6451  &       -2118324 & 4897   &    -0 & 14 \\ \hline
32  &     1461214840.52951   &    1500568329.7155    &      -2 & 6226  &       -1780722 & 8958   &    -0 & 12 \\ \hline
33  &     1460597607.61728   &    1500568329.7155    &      -2 & 6637  &       -2397955 & 8080   &    -0 & 16 \\ \hline
34  &     1462056933.32981   &    1500568329.7155    &      -2 & 5665  &        -938630 & 0955   &    -0 & 06 \\ \hline



\end{tabular}
\end{center}
\end{table}


\subsection{Opad wymierny}


\chapter{Podsumowanie}

Celem pracy była analiza danych opadowych zamiarem oszacowania ilości wody jaka spadła na zadany obszar, co przeprowadzane odpowiednio wcześnie mogłoby być sygnałem dla rozpoczęcia procesu reagowania przeciwpowodziowego. Przeprowadzono przegląd istniejących metod wyznaczania opadu powierzchniowego na podstawie pomiarów punktowych, a także możliwości dostępu do takich pomiarów. Znaleziono oraz przetestowano metodę, która pozwoliła uzyskać zadowalające rezultaty bazując na danych symulowanych.

\section{Możliwe rozszerzenia}
Oczywiście zastosowane rozwiązanie można poddawać rozszerzeniu. Przede wszystkim należy wspomnieć o~zastosowaniu rzeczywistych pomiarów z~istniejących posterunków opadowych. Wówczas rezultaty mogłyby być bardziej miarodajne. Pamiętać jednak należy o~nawet kilkumiesięcznym okresie oczekiwania na udzielenie do nich dostępu.

Odwołując się do prawdziwych pomiarów, niniejszy projekt można rozszerzyć o~uwzględnianie ukształtowania terenu wyznaczając tym sposobem kierunek lub nawet prędkość przesuwania się mas wodnych w~kontekście ochrony przeciwpowodziowej lub systemów alarmowania.

Innym usprawnieniem może być użycie bardziej zaawansowanej metody konwersji współrzędnych geograficznych. Zaproponować można chociażby wyznaczanie współczynnika konwersji (metrycznej długości jednego stopnia długości geograficznej na zadanej szerokości geograficznej) osobno dla każdego z~analizowanych punktów. W przypadku analiz rzeczywistych posterunków zmiana ta może być wielce istotna, nie powodująca zakłamań w~powierzchniach zadanego terenu.
\chapter{Słownik pojęć}
\begin{description}[leftmargin=6cm]

\item[Posterunek opadowy]
Miejsce przeprowadzania pomiarów i obserwacji meteorologicznych. Mierzona jest ilość wody opadowej, która zebrała się w~deszczomierzu.

\item[Posterunek wodowskazowy]
Miejsce prowadzenia pomiarów stanu wody za pomocą wodowskazu.

\item[Profil wodowskazowy]
Punkt na rzece, w~którym zamontowany jest wodowskaz.

\item[Zlewnia]
Obszar, z którego wody spływają do jednego punktu danego zbiornika wodnego lub jego fragmentu.

\item[Zbiornik retencyjny]
Sztuczny zbiornik wodny powstały w wyniku zatamowania wód rzecznych przez zaporę. Pełnić może wiele funkcji, takich jak przeciwpowodziowa energetyczna czy nawet rekreacyjna.

%\item[Opad]
%Mierzony punktowo w~miejscu posterunku opadowego z~użyciem deszczomierzy.

\item[Izohieta]
Linia łącząca punkty o~jednakowej wysokości opadu.

\item[Triangulacja]
Zagadnienie geometrii obliczeniowej. Metoda łączenia zbioru punktów w~sieć trójkątów (przestrzeń dwuwymiarowa) bądź czworościanów (przestrzeń trójwymiarowa) o~wierzchołkach w~zadanych punktach. Sposób rozkładu kształtu na mniej skomplikowane elementy.
\end{description}
\chapter{Dodatek A - fragmenty kodów źródłowych}
\label{cha:dodatek_A}
\section{Odczyt danych z serwisu}

\begin{lstlisting}
function z = ellipsoidalPrecip(x, y)
global middle_x;
global middle_y;

x0 = middle_x;
y0 = middle_y;
% x0 = 2;  y0 = 4;
h = 100;
alpha = h/(300000000000); 
betha = h/(40000000000);

z = h - alpha .* (x-x0).^2 - betha.*(y-y0).^2;
[w,k] = size(z);
for i=1:w,
    for j=1:k,
        if z(i,j) < 0, z(i,j)=0; end
    end
end
end
\end{lstlisting}


\label{sec:kod_odczyt_danych}
\section{Interpolacja}
\section{inny podrozdzial}


% \include{tests}
\bibliographystyle{abbrv}
\bibliography{bibliografia}

\end{document}
