\chapter{Użyte technologie}
\label{cha:użyte technologie}

Tak jak opisano w sekcji \ref{sec:cele pracy} cele pracy będzie zaimplementowanie detektora ruchu na system Android, który będzie łączył się z serwerem i przekazywał tam pewne informacje o swojej pracy. Nie trudno wyciągnąć wniosek, że do pracy potrzebne będą następujące komponenty:

\begin{itemize}
	\item urządzenie z system Android wyposażone w kamerę plus Android SKD,
	\item serwer na którym będzie ``postawiona'' aplikacja z którą komunikować się będzie smartfon; wybór padł na Amazon AWS,
	\item system do zarządzania bazami danych,
	\item ponieważ aplikacja webowa będzie pisana w języku Java przydatny staje się serwer aplikacji. Wybrany został Apache Tomcat. 	
\end{itemize}

Komunikacja pomiędzy smartfonem a aplikacją webową będzie odbywać się przy użyciu architektury REST oraz formatu danych JSON. 

\section{Android OS}
\label{sec:android}

Android jest system operacyjnym o jądro Linuxa. Jest to najpopularniejszy system operacyjny na urządzenia mobilne (smartfony i tablety). 

W kontekście tej pracy najważniejszą rzeczą o której trzeba powiedzieć jest fakt, iż natywne aplikacje na ten system tworzy się w używając Android SDK który udostępnia API do języka Java. Dokumentacja znajduje się na oficjalnej stronie projektu Android \cite{Androiddocs}.

Jest to bardzo ważne ponieważ kod tworzony przez developerów, tak samo jak przy korzystaniu ze ``zwykłej'' Javy, jest kompilowany do kodu bajtowego. Kod ten nie jest rozumiany przez procesor, a do jego uruchomienia potrzebna jest maszyna wirtualna. W przypadku Androida jest to Dalvik, który został zastąpiony w najnowszej wersji systemu Android Lolipop maszyną wirtualną ART.

Kod uruchamiany w maszynie wirtualnej jest wolniejszy od kodu maszynowego. W przypadku prostych użytkowych aplikacji, które nie wykonują skomplikowanych obliczeń fakt ten nie przeszkadza istotnie, jednak kiedy piszemy program który wykonuje dużo obliczeń sprawa się komplikuje. Kod wykonuje się dużo wolniej i zużywa więcej pamięci. Z tego powodu implementowanie algorytmu detekcji ruchu w Javie na urządzeniach mobilnych mija się z celem. Takie rozwiązanie jest całkowicie bezużyteczne. Na szczęście inżynierowie odpowiadający za Androida przewidzieli taką sytuację i udostępnili coś co nazywa się NDK -- {\it Native Development Kit}. Jest to zestaw narzędzi umożliwiających programowanie urządzeń z Androidem w języku C/C++. NDK umożliwia (a nawet wymaga) integracje z SDK. 

W praktyce NDK używa się do programowania tylko tych części aplikacji które są krytyczne pod względem wydajnościowym. W przypadku tej pracy -- części aplikacji analizującej kolejne klatki z aparatu. 

Trzeba też zaznaczyć że programowanie z użyciem NDK jest znacznie mniej wygodne niż SDK. Na szczęście istnieje biblioteka {\it OpenCV -- Open source Computer Vision} \cite{Opencv}. Jest ona przeznaczona dla systemów przetwarzających obrazy. Udostępnia API Javo-we i jest przygotowana do pracy z system Android. Biblioteka ta została szerzej opisana w sekcji \ref{sec:biblioteki zewnętrzne} 

%\section{Serwer: Amazon AWS}
%\label{sec:serwer}
%
%Amazon AWS -- tutaj postawiłem bazę danych i tomcata. Po krótce opiszę serwer. Raczej krótka %sekcja 

\section{Baza danych: MySQL}
\label{sec:baza danych}

MySQL \cite{Mysql} jest serwerem oraz systemem zarządzania relacyjnymi bazami danych. Dostęp do bazy danych uzyskujemy poprzez język zapytań bazy danych SQL. MySQL jest rozwijany przez firmę Oracle. 

Relacyjna baza danych jest niczym innym jak zbiorem relacji. Przedstawiając sprawę jak najprościej można powiedzieć, że relacja jest dwuwymiarową tabelą złożoną z kolumn i wierszy, gdzie liczba kolumn jest z góry określona. Każdą kolumnę określa nazwa, oraz dziedzina. Na przecięciu wiersza i kolumny znajduje się pojedyncza wartość należąca do dziedziny kolumny. Wiersze w relacji są unikalne, tj. nie nie występuje taka para wierszy że wartości dla każdej z kolumn są takie same. 
Kolejność wierszy i kolumn jest nieistotna. 


\section{Serwer aplikacji: Tomcat}
\label{sec:serwer aplikacji}

Ponieważ aplikacja serwerowa pisana jest w języku Java EE niezbędny jest serwer aplikacji. {\it Apache Tomcat} \cite{Tomcat} idealnie się do tego nadaje. Tomcat jest otwarto--źródłową implementacją technologii {\it Java Servlet}, oraz {\it JavaServer Pages}. 

\section{Forma komunikacji: REST i Jersey}
\label{sec:forma komunikacji}

{\it REST -- Representational State Transfer} \cite{Rest} -- jest to abstrakcyny wzorzec architektury oprogramowania opisujący sposób komunikacji aplikacji rozproszonych. Jest bezstanowy, to znaczy, że klient musi dostarczyć wszelkich niezbędnych informacji do realizacji zapytania.

{\it JAX-RS: Java API for RESTful Web Services} jest API Javowym zapewniającym wsparcie w tworzeniu usług internetowych ({\it web service}) zgodnych z wzrocem REST. {\it JAX-RS} jest częścią Javy EE 6.

{\it Rest Jersey} \cite{Jersey} jest otwarto--źródłowym framework'iem wspomagającym tworzenie webservice'ów RESTowych. {\it Jersey} w pełni wspiera {\it JAX-RS}. 

W pracy aplikacja kliencka komunikuje się z serwerem przy użyciu formatu {\it JSON} \cite{Json}. Jest to lekki, tekstowy i nadający się do czytania przez człowieka format wymiany danych. Przykładowe dane przekazywane w tym formacie: 

\begin{lstlisting}
{"obiad": {
   "kalorie": 800,
   "godzina": "14:00",
   "składniki": [
     {
		"nazwa": "schabowy",
		"kalorie": 600
	 },
	 {
		"nazwa": "ziemniaki",
		"kalorie": 200
   ]
 }}
\end{lstlisting}
