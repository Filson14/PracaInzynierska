\chapter{Dodatek A - fragmenty kodów źródłowych}
\label{cha:dodatekA}

\section{Android}
\label{sec:android}


\subsection{Klient RESTowy}
\label{sec:androidApi}

Kod klienta 

\begin{lstlisting}
package pl.edu.agh.misztal.secam.rest;

import pl.edu.agh.misztal.secam.rest.callback.LoginCallback;
import pl.edu.agh.misztal.secam.rest.model.EmptyEntity;
import pl.edu.agh.misztal.secam.rest.model.Login;
import retrofit.Callback;
import retrofit.http.DELETE;
import retrofit.http.Field;
import retrofit.http.FormUrlEncoded;
import retrofit.http.Header;
import retrofit.http.Multipart;
import retrofit.http.POST;
import retrofit.http.Part;
import retrofit.mime.TypedFile;

public interface SecamApi {

    @FormUrlEncoded
    @POST("/detector/login")
    void login(@Header("api_key") String apiKey, @Field("email") String email, @Field("password") String password, @Field("device_id") String deviceId, Callback<Login> callback);

    @DELETE("/users/logout")
    void logout(@Header("api_key") String apiKey, @Header("token") String token, Callback<EmptyEntity> callback);

    @FormUrlEncoded
    @POST("/detector/devicename")
    void setDevicename(@Header("api_key") String apiKey, @Header("token") String token, @Field("device_id") String deviceId, @Field("name") String name, Callback<EmptyEntity> callback);

    @FormUrlEncoded
    @POST("/detector/beep")
    void beep(@Header("api_key") String apiKey, @Header("token") String token, @Field("device_id") String deviceId, Callback<EmptyEntity> callback);

    @Multipart
    @POST("/detector/photo")
    void postPhoto(@Header("api_key") String apiKey, @Header("token") String token, @Part("device_id") String deviceId, @Part("photo")TypedFile photo, Callback<EmptyEntity> callback);

}
\end{lstlisting}

Tworzenie obiektu będącego klientem:

\begin{lstlisting}

// Tworzenie GSON'a, tak aby poprawnie automatycznie konwertował wiadomość JSON na obiekt POJO ({\it Plain Old Java Object})
Gson gson = new GsonBuilder()
                    .setFieldNamingPolicy(FieldNamingPolicy.UPPER_CAMEL_CASE)
                    .create();
			
			// Ustawienie konwertera i adresu serwera
            RestAdapter adapter = new RestAdapter.Builder()
                    .setConverter(new GsonConverter(gson))
                    .setEndpoint(ENDPOINT)
                    .build();

            SecamApi api = adapter.create(SecamApi.class);

\end{lstlisting}

\subsection{Wykrywanie ruchu}

\begin{lstlisting}

curr.copyTo(prev);
next.copyTo(curr);
inputFrame.gray().copyTo(next);
binarizeMovement();
			
// (...)
			
if(isMovement())
		reactMovement(in);

\end{lstlisting}

Po wykoananiu {\it binarizeMovement()} w zmiennej {\it result} mamy binarną macierz z zanzaczonymi pikselami które się zmieniły. 

Oto kod funkcji {\it isMovement()} stwierdzającej czy ruch wystąpił:

\begin{lstlisting}
private boolean isMovement() {
	//sprawdź czy jest minimalna ilość zmienionych pikseli
	if(result != null && Core.countNonZero(result) > preferences.getMinimumMovement()) {
		if(preferences.isUseStdDev()) { //czy bierzemy pod uwagę odchylenie standardowe ?
			MatOfDouble mean = new MatOfDouble();
			MatOfDouble stddev = new MatOfDouble();
			Core.meanStdDev(result, mean, stddev);
			
			if(stddev.get(0, 0)[0] < preferences.getMaxStdDev()) 
				return true;
			else
				return false;
		}
		return true;	
	}
	
	return false;
}

\end{lstlisting}
			
\section{Serwer API}
\label{sec:serwerApi}

Implementacja klasy odpowiadającej za udostępnienie API

\begin{lstlisting}
@Path("v1.0")
public class Api {
	
	private static final String apiKey = "asd";
	
	private static final int MINUTES_ONLINE = 5; //minutes in the past that beeps is considered to be online
	
	private DBController db = new DBController();
	private String baseImagePath = "/var/www/images/secam/";
	//private String baseImagePath = "F:/SeCam/";
	private Gson gson = new GsonBuilder()
		.setFieldNamingPolicy(FieldNamingPolicy.UPPER_CAMEL_CASE)
		.excludeFieldsWithoutExposeAnnotation()
		.create();
	
	// USERS ============================================================
	
	@POST
	@Path("users/register")
	public Response register(@HeaderParam("api_key") String apiKey, @FormParam("email") String email, @FormParam("password") String password) {
		Response ret;
		if(!apiKey.equals(Api.apiKey)) {
			ret = Response.status(Response.Status.UNAUTHORIZED).build();
		} else {
			boolean ok = db.newUser(email, password);
			if(ok) {
				ret = Response.ok().build();
			} else {
				Error error = new Error("User register", "User exists");
				ret = Response.status(Response.Status.NOT_ACCEPTABLE).entity(gson.toJson(error)).build();
			}
		}
		return ret;
	}
	
	@POST
	@Produces(MediaType.APPLICATION_JSON)
	@Path("/users/login")
	public Response login(@HeaderParam("api_key") String apiKey, @FormParam("email") String email, @FormParam("password") String password) {
		Response ret;
		if(!apiKey.equals(Api.apiKey)) {
			ret = Response.status(Response.Status.UNAUTHORIZED).build();
		} else {
			String token = db.newSession(email, password, "client");
			if(token != null) {
				ret = Response.ok(gson.toJson(new Token(token))).build();
			} else {
				Error error = new Error("User login", "Incorrect credentials");
				ret = Response.status(Response.Status.NOT_ACCEPTABLE).entity(gson.toJson(error)).build();
			}
		}
		
		return ret;
	}
	
	@DELETE
	@Path("users/logout")
	public Response logout(@HeaderParam("api_key") String apiKey, @HeaderParam("token") String token) {
		Response ret;
		
		if(!apiKey.equals(Api.apiKey)) {
			ret = Response.status(Response.Status.UNAUTHORIZED).build();
		} else {
			boolean ok = db.deleteSession(token);
			
			if(ok)
				ret = Response.ok().build();
			else
				ret = Response.status(Response.Status.UNAUTHORIZED).build();
		}
		
		
		
		return ret;		
	}
	
	// DETECTOR ============================================================
	
	
	@POST
	@Produces(MediaType.APPLICATION_JSON)
	@Path("/detector/login")
	public Response detectorLogin(@HeaderParam("api_key") String apiKey, @FormParam("email") String email, @FormParam("password") String password,
			@FormParam("device_id") String deviceId) {
		Response ret;
		if(!apiKey.equals(Api.apiKey)) {
			ret = Response.status(Response.Status.UNAUTHORIZED).build();
		} else {
			String token = db.newSession(email, password, "detector");
			if(token != null) {
				int uid = db.getUserIdByToken(token);
				if(db.getDeviceIdByImei(deviceId, uid) < 0) { //dodaj urządzenie jeżeli nie istnieje
					boolean ok = db.newDevice(uid, deviceId);
					if(ok)
						ret = Response.ok(gson.toJson(new Token(token))).build();
					else
						ret = Response.status(Response.Status.INTERNAL_SERVER_ERROR).build();
				}
				else
					ret = Response.ok(gson.toJson(new Token(token))).build();
			} else {
				Error error = new Error("User login", "Incorrect credentials");
				ret = Response.status(Response.Status.NOT_ACCEPTABLE).entity(gson.toJson(error)).build();
			}
		}
		
		return ret;
	}
	
	@POST
	@Path("detector/beep")
	public Response beep(@HeaderParam("api_key") String apiKey, @HeaderParam("token") String token, @FormParam("device_id") String deviceId) {
		Response ret;
		int uid = db.getUserIdByToken(token);
		
		if(!apiKey.equals(Api.apiKey) || uid < 0) {
			ret = Response.status(Response.Status.UNAUTHORIZED).build();
		} else {
			int _deviceid = db.getDeviceIdByImei(deviceId, uid);
			if(_deviceid >= 0) {
				boolean ok = db.beep(_deviceid);
				if(ok)
					ret = Response.ok().build();
				else
					ret = Response.status(Response.Status.INTERNAL_SERVER_ERROR).build();
			} else 
				ret = Response.status(Response.Status.NOT_FOUND).build();
		}
		
		return ret;
	}
	
	@POST
	@Path("detector/devicename")
	public Response setDeviceName(@HeaderParam("api_key") String apiKey, @HeaderParam("token") String token, @FormParam("device_id") String deviceId,
			@FormParam("name") String name) {
		Response ret;
		
		int uid = db.getUserIdByToken(token);
		
		if(!apiKey.equals(Api.apiKey) || uid < 0) {
			ret = Response.status(Response.Status.UNAUTHORIZED).build();
		} else {
			int _deviceid = db.getDeviceIdByImei(deviceId, uid);
			if(_deviceid < 0)
				ret = Response.status(Status.NOT_FOUND).build();
			else {
				boolean ok = db.setDeviceName(_deviceid, name);
				if(ok)
					ret = Response.ok().build();
				else
					ret = Response.status(Status.INTERNAL_SERVER_ERROR).build();
			}
		}
		
		return ret;
	}
	
	@POST
	@Path("detector/photo")
	@Consumes(MediaType.MULTIPART_FORM_DATA)
	public Response postPhoto(@HeaderParam("api_key") String apiKey, @HeaderParam("token") String token, @FormDataParam("device_id") String deviceId,
			@FormDataParam("photo") InputStream fileInputStream) {
		
		Response ret;
		
		int uid = db.getUserIdByToken(token);
		
		if(!apiKey.equals(Api.apiKey) || uid < 0) {
			ret = Response.status(Response.Status.UNAUTHORIZED).build();
		} else {
			int _deviceid = db.getDeviceIdByImei(deviceId, uid);
			if(_deviceid < 0)
				ret = Response.status(Status.NOT_FOUND).build();
			else {
				File dir = new File(baseImagePath + Integer.toString(uid) + "/" + Integer.toString(_deviceid));
				dir.mkdirs();
				int imageCount = db.getImages(_deviceid).size();
				String filepath = dir.getAbsolutePath() + "/" + Integer.toString(imageCount) + ".jpg";
				
				if(writeToFile(fileInputStream, filepath)){
					boolean ok = db.addImage(_deviceid, filepath);
					if(ok)
						Response.ok().build();
					else
						Response.status(Status.INTERNAL_SERVER_ERROR);
				}
				ret = Response.ok().build();
			}
		}
		
		return ret;
		
	}
	
	// CLIENT ===========================================================
	
	@GET
	@Path("client/devices")
	@Produces(MediaType.APPLICATION_JSON)
	public Response getDevices(@HeaderParam("api_key") String apiKey, @HeaderParam("token") String token) {
		Response ret;
		
		int uid = db.getUserIdByToken(token);
		
		if(!apiKey.equals(Api.apiKey) || uid < 0) {
			ret = Response.status(Response.Status.UNAUTHORIZED).build();
		} else {
			List<Devices> list = db.getDevices(uid);
			if(list != null) {
				DevicesList devicesList = new DevicesList(list);
				ret = Response.ok(gson.toJson(devicesList)).build();
				
			} else 
				ret = Response.status(Status.NOT_FOUND).build();
		}
		
		return ret;
		
	}
	
	@GET
	@Path("client/isonline")
	@Produces(MediaType.APPLICATION_JSON)
	public Response isOnline(@HeaderParam("api_key") String apiKey, @HeaderParam("token") String token, @QueryParam("device_id") String deviceId) {
		Response ret;
		
		int uid = db.getUserIdByToken(token);
		
		if(!apiKey.equals(Api.apiKey) || uid < 0) {
			ret = Response.status(Response.Status.UNAUTHORIZED).build();
		} else {
			Beeps b = db.getBeeps(deviceId, uid);
			IsOnline online;
			
			if(b != null) { 
				if(isOnline(b.getTimestamp()))
					online = new IsOnline(true);
				else
					online = new IsOnline(false);
				
				ret = Response.ok(gson.toJson(online)).build();
			} else
				ret =  Response.status(Status.NOT_FOUND).build();
		}
		
		return ret;
		
	}
	
	
	@GET
	@Path("client/images")
	@Produces(MediaType.APPLICATION_JSON)
	public Response getImages(@HeaderParam("api_key") String apiKey, @HeaderParam("token") String token, @QueryParam("device_id") String deviceId) {
		Response ret;
		
		int uid = db.getUserIdByToken(token);
		
		if(!apiKey.equals(Api.apiKey) || uid < 0) {
			ret = Response.status(Response.Status.UNAUTHORIZED).build();
		} else {
			int _deviceid = db.getDeviceIdByImei(deviceId, uid);
			if(_deviceid < 0)
				ret= Response.status(Status.NOT_FOUND).build();
			else {
				List<Images> list = db.getImages(_deviceid);
				ImagesList images = new ImagesList(list);
				
				ret = Response.ok(gson.toJson(images)).build();
			}
		}
			
		
		return ret;
		
	}
	
	@GET
	@Path("client/images/fetch/{id}")
	@Produces("image/jpeg")
	public Response getImage(@HeaderParam("api_key") String apiKey, @HeaderParam("token") String token, @PathParam("id") String _id, @QueryParam("device_id") String deviceId) {
		Response ret;
		
		int uid = db.getUserIdByToken(token);
		
		if(!apiKey.equals(Api.apiKey) || uid < 0) {
			ret = Response.status(Response.Status.UNAUTHORIZED).build();
		} else {
			try {
				int id = Integer.parseInt(_id);
				int _deviceid = db.getDeviceIdByImei(deviceId, uid);
				if(_deviceid < 0)
					ret = Response.status(Status.NOT_FOUND).build();
				else {
					String filepath = baseImagePath + "/" + Integer.toString(uid) + "/" + Integer.toString(_deviceid) + "/"
							+ Integer.toString(id) + ".jpg";
					
					File file = new File(filepath);
					
					if(file.exists()) {
						ret = Response.ok(file, "image/jpeg").build();
					} else {
						ret = Response.status(Response.Status.NOT_FOUND).build();
					}
				} 
			} catch(NumberFormatException ex) {
				ret = Response.status(Response.Status.BAD_REQUEST).build();
			}
		}
		
		return ret;
	}
	
	// PRIVATE  ============================================================
	
	private boolean writeToFile(InputStream uploadedInputStream,
			String uploadedFileLocation) {
	 
			try {
				OutputStream out = new FileOutputStream(new File(
						uploadedFileLocation));
				int read = 0;
				byte[] bytes = new byte[1024];
	 
				out = new FileOutputStream(new File(uploadedFileLocation));
				while ((read = uploadedInputStream.read(bytes)) != -1) {
					out.write(bytes, 0, read);
				}
				out.flush();
				out.close();
			} catch (IOException e) {
				e.printStackTrace();
				return false;
			}
			return true;
	 
		}
	
	private boolean isOnline(Timestamp t) {
		long milis = (new Date()).getTime();
		Date border = new Date(milis - MINUTES_ONLINE * 60 * 1000);
		if(t.before(border))
			return false;
		
		return true;
	}
	
}
\end{lstlisting}


\section{Baza danych - kod SQL}
\label{sec:dbsql}

	\lstset{
		language=SQL}
		
	\begin{lstlisting}
	
USE secam;

SET FOREIGN_KEY_CHECKS=0;


--  Drop Tables, Stored Procedures and Views 

DROP TABLE IF EXISTS Beeps CASCADE;
DROP TABLE IF EXISTS Devices CASCADE;
DROP TABLE IF EXISTS Images CASCADE;
DROP TABLE IF EXISTS Sessions CASCADE;
DROP TABLE IF EXISTS Users CASCADE;

--  Create Tables 
CREATE TABLE Beeps
(
	id INTEGER UNSIGNED NOT NULL AUTO_INCREMENT,
	deviceid INTEGER UNSIGNED NOT NULL,
	timestamp TIMESTAMP NOT NULL DEFAULT CURRENT_TIMESTAMP,
	PRIMARY KEY (id),
	UNIQUE UQ_Beeps_deviceid(deviceid),
	UNIQUE UQ_Beeps_id(id),
	KEY (deviceid)

) ;


CREATE TABLE Devices
(
	id INTEGER UNSIGNED NOT NULL AUTO_INCREMENT,
	userid INTEGER UNSIGNED NOT NULL,
	deviceid VARCHAR(50) NOT NULL,
	devicename VARCHAR(50),
	PRIMARY KEY (id),
	UNIQUE UQ_Devices_id(id),
	UNIQUE UQ_PAIR_Deviceid_Userid(deviceid, userid),
	KEY (userid)

) ;


CREATE TABLE Images
(
	id INTEGER UNSIGNED NOT NULL AUTO_INCREMENT,
	deviceid INTEGER UNSIGNED NOT NULL,
	path VARCHAR(100) NOT NULL,
	timestamp TIMESTAMP NOT NULL DEFAULT CURRENT_TIMESTAMP,
	PRIMARY KEY (id),
	UNIQUE UQ_Images_id(id),
	KEY (deviceid)

) ;


CREATE TABLE Sessions
(
	id INTEGER UNSIGNED NOT NULL AUTO_INCREMENT,
	userid INTEGER UNSIGNED NOT NULL,
	token VARCHAR(50) NOT NULL,
	sessiontype VARCHAR(10) NOT NULL,
	expires DATETIME,
	PRIMARY KEY (id),
	UNIQUE UQ_Sessions_id(id),
	UNIQUE UQ_Sessions_token(token),
	KEY (userid)

) ;


CREATE TABLE Users
(
	id INTEGER UNSIGNED NOT NULL AUTO_INCREMENT,
	email VARCHAR(40) NOT NULL,
	password VARCHAR(40) NOT NULL,
	PRIMARY KEY (id),
	UNIQUE UQ_Users_email(email),
	UNIQUE UQ_Users_id(id)

) ;



SET FOREIGN_KEY_CHECKS=1;


--  Create Foreign Key Constraints 
ALTER TABLE Beeps ADD CONSTRAINT FK_Beeps_Devices 
	FOREIGN KEY (deviceid) REFERENCES Devices (id);

ALTER TABLE Devices ADD CONSTRAINT FK_Devices_Users 
	FOREIGN KEY (userid) REFERENCES Users (id);

ALTER TABLE Images ADD CONSTRAINT FK_Images_Devices 
	FOREIGN KEY (deviceid) REFERENCES Devices (id);

ALTER TABLE Sessions ADD CONSTRAINT FK_Sessions_Users 
	FOREIGN KEY (userid) REFERENCES Users (id);






--  Create Triggers 
DROP TRIGGER IF EXISTS tr_setexpiredate;

DELIMITER $$
CREATE TRIGGER tr_setexpiredate BEFORE INSERT ON Sessions
FOR EACH ROW
BEGIN
	SET NEW.expires = DATE_ADD(NOW(), INTERVAL 280 DAY);
END $$
;


\end{lstlisting}