\chapter{Podsumowanie}
\label{cha:podsumowanie}

Zrealizowany projekt składa się z dwóch modułów ze sobą współpracujących: aplikacji na system Android, oraz aplikacji serwerowej udostępniającej restowe API. Obie te części realizują wszystkie założenia projektu. Korzystając z rezultatów tej pracy otrzymujemy narzędzie które:

\begin{itemize}
	\item umożliwia automatyczną detekcję ruchu,
	\item zaznacza wykryty ruch na klatce, 
	\item jest w stanie powiadomić o wykrytym ruchu poprzez alarm dźwiękowy,
	\item powiadamia serwer o tym, że pracuje, 
	\item powiadamia serwer o wykrytym ruchu, 
	\item wysyła na serwer klatkę z wykrytym ruchem,
	\item zapisuje na karcie pamięci klatki z wykrytym ruchem.
\end{itemize}

Dzięki temu dostajemy narzędzie, które może zastępować prostą kamerę bezpieczeństwa z możliwością zdalnego oglądania rezultatów jej pracy.

\section{Perspektywy dalszego rozwoju}

Praca jak najbardziej nadaje się do dalszego rozwoju. Nad projektem można pracować na dwóch płaszczyznach:
\begin{itemize}
	\item ulepszania algorytmu detekcji ruchu, 
	\item powiększania ilości sposobów na komunikację z użytkownikiem.
\end{itemize}

W pierwszym przypadku możliwe jest napisanie szybszej i dokładniejszej detekcji. Dzięki temu, poza precyzją, można osiągnąć większą energooszczędność. 

Drugi przypadek jest zdecydowanie bardziej nastawiony bezpośrednio na {\it user experience}. Możliwe jest dodanie streamingu wideo obserwowanej sceny, co umożliwi podgląd na żywo tego co dzieje się przed kamerą. Wraz z dodaniem nowych metod komunikacji użytkownika o wykrytym ruchu, takich jak wiadomość SMS i MMS, projekt może z powodzeniem zastępować nawet bardziej zaawansowane kamery bezpieczeństwa. 

