\chapter{Wstęp}
\label{cha:wstęp}

\section{Przedmowa}
\label{sec:przedmowa}

Ostatnie lata były czasem gwałtownego rozpowszechnienia się smartfonów -- urządzeń łączących cechy telefonu komórkowego, oraz komputera osobistego. W roku 2011 przewidywana sprzedaż smartfonów wzrosła o 175\% \cite{Wwwsales}. Urządzenia te posiadają coraz więcej możliwości i są w stanie zastępować m.in. kamery, aparaty cyfrowe, odtwarzacze wideo, konsole do gier i wiele innych. Sam współtwórca wskaźnika {\it Smartphone Index} na amerykańskiej giełdzie NASDAQ powiedział: ``Nieustanna ewolucja smartfonów i tabletów generuje nowe możliwości ich wykorzystania, zwiększając tym samym ich użyteczność i wartość, ale jednocześnie wypierając inne urządzenia z polskich domów, np. aparaty cyfrowe''. 

Idąc tym tropem pomysł zastąpienia kamery bezpieczeństwa smartfonem wydaje się bardzo trafiony. Rozwiązanie to umożliwia łatwe implementowanie i rozszerzanie funkcjonalności. Dodatkowo ludzie często posiadają swoje stare smartfony, przez co zakup kamery bezpieczeństwa może sprowadzać się do zainstalowania odpowiedniej aplikacji. 

Praca ta będzie opisywać to zagadnienie zarówno od strony teoretycznej jak i implementacyjnej.

\section{Opis dokumentu}
\label{sec:opis dokumentu}

Praca ta traktuje o użyciu smartfona jako kamery bezpieczeństwa z zaimplementowanym mechanizmem detekcji ruchu. Opisuje teorię wykrywania ruchu oraz technologię wykorzystaną do implementacji aplikacji na system {\it Android OS} jak i serwera z którą aplikacja ta będzie wymieniać dane. 

W pracy zawarto rozdziały:
\begin{description}
	\item[Algorytm detekcji ruchu] w rozdziale opisano użyty algorytm wykrywania ruchu, oraz ogarniczenia jakie wprowadza odpalanie algorytmu na smartfonie,
	\item[Użyte technologie] opisane zostały tu wszystkie technologie, biblioteki oraz framework-i wykorzystane w pracy. Uzasadniony został też ich wybór,
	\item[Implementacja] opisane problemy i rozwiązania części implementacyjnej systemu. Opisane zostało też API z którego korzysta aplikacja. 
\end{description}

\section{Cele pracy}
\label{sec:cele pracy}

Celem pracy jest stworzenie automatycznego detektora ruchu na system {\it Android OS}. Aplikacja ta będzie w stanie:
\begin{itemize}
	\item wykrywać ruch, oraz zaznaczać na obrazie fragment sceny gdzie ten ruch został wykryty,
	\item powiadomić użytkownika o wykrytym ruchu poprzez alarm dźwiękowy,
	\item powiadomić serwer o tym, że jest detekcja ruchu jest włączona,
	\item zapisać oraz przesłać na serwer klatki z wykrytym ruchem.
\end{itemize}

Stworzony serwer będzie w stanie przechowywać informacje o stanie aplikacji -- o tym czy włączona jest detekcja ruchu. Dodatkowo będzie przechowywał klatki zawierające wykryty ruch. Użytkownik poprzez prostą stronę www będzie w stanie zalogować się oraz oglądać logi z pracy swojego detektora.